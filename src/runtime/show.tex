\nwfilename{show.nw}\nwbegindocs{0}% -*- noweb-code-mode: c-mode -*-% ===> this file was generated automatically by noweave --- better not edit it
% $Id: show.nw,v 2.16 2004/05/01 13:18:14 wlux Exp $
%
% Copyright (c) 2001-2004, Wolfgang Lux
% See ../LICENSE for the full license.
%
\subsection{Show Function}
The function \verb|show :: a -> String| is used to convert a value
lazily into a string. This function uses the information available in
the node info structure in order to determine how its argument should
be displayed. A minor variant of \verb|show| is the impure function
\verb|dval :: a -> String| which neither does evaluate its argument
nor wait until unbound variables are instantiated. The primary
objective of this function is to support the declarative debugger.

\nwenddocs{}\nwbegincode{1}\sublabel{NW4bhQ2K-3k7b8Q-1}\nwmargintag{{\nwtagstyle{}\subpageref{NW4bhQ2K-3k7b8Q-1}}}\moddef{show.c~{\nwtagstyle{}\subpageref{NW4bhQ2K-3k7b8Q-1}}}\endmoddef\nwstartdeflinemarkup\nwprevnextdefs{\relax}{NW4bhQ2K-3k7b8Q-2}\nwenddeflinemarkup
#include "config.h"
#include <stdio.h>
#include <stdlib.h>
#include <string.h>
#include <math.h>
#include "debug.h"
#include "run.h"
#include "heap.h"
#include "stack.h"
#include "threads.h"
#include "spaces.h"
#include "trail.h"
#include "eval.h"
#include "vars.h"
#include "char.h"
#include "cstring.h"
#include "cam.h"
#include "trace.h"

DECLARE_ENTRYPOINT(__show);
DECLARE_ENTRYPOINT(__dval);
DECLARE_LABEL(showp);
DECLARE_LABEL(showp_eval);
DECLARE_LABEL(showp_lazy);
DECLARE_LABEL(showp_1);
DECLARE_LABEL(showArgs);
DECLARE_LABEL(showArgs_eval);
DECLARE_LABEL(showArgs_lazy);
DECLARE_LABEL(showInfix);
DECLARE_LABEL(showTuple);
DECLARE_LABEL(showTupleArgs);
DECLARE_LABEL(showTupleArgs_eval);
DECLARE_LABEL(showTupleArgs_lazy);
DECLARE_LABEL(showList);
DECLARE_LABEL(showTail);
DECLARE_LABEL(showTail_eval);
DECLARE_LABEL(showTail_lazy);
DECLARE_LABEL(showTail_1);
DECLARE_LABEL(showString);
DECLARE_LABEL(showStringTail);
DECLARE_LABEL(showStringTail_eval);
DECLARE_LABEL(showStringTail_lazy);

#if ONLY_BOXED_OBJECTS
static struct int_node zero_node = \{ &int_info, 0 \};
static struct int_node one_node  = \{ &int_info, 1 \};
static struct int_node two_node  = \{ &int_info, 2 \};

# define zero (Node *)&zero_node
# define one  (Node *)&one_node
# define two  (Node *)&two_node
#else
# define zero mk_int(0)
# define one  mk_int(1)
# define two  mk_int(2)
#endif

#define space    (Node *)(char_table + ' ')
#define comma    (Node *)(char_table + ',')
#define lparen   (Node *)(char_table + '(')
#define rparen   (Node *)(char_table + ')')
#define lbracket (Node *)(char_table + '[')
#define rbracket (Node *)(char_table + ']')
#define bar      (Node *)(char_table + '|')
#define dblquote (Node *)(char_table + '\\"')

static FunctionInfo showp_info = FUNINFO("showp", showp, 4);
static FunctionInfo showArgs_info = FUNINFO("showArgs", showArgs, 4);
static FunctionInfo showTupleArgs_info = 
    FUNINFO("showTupleArgs", showTupleArgs, 3);
static FunctionInfo showTail_info = FUNINFO("showTail", showTail, 3);
static FunctionInfo showStringTail_info =
    FUNINFO("showStringTail", showStringTail, 2);

static NodeInfo showp_suspend_info = SUSPINFO(showp);
static NodeInfo showArgs_suspend_info = SUSPINFO(showArgs);
static NodeInfo showTupleArgs_suspend_info = SUSPINFO(showTupleArgs);
static NodeInfo showTail_suspend_info = SUSPINFO(showTail);
static NodeInfo showStringTail_suspend_info = SUSPINFO(showStringTail);

\nwalsodefined{\\{NW4bhQ2K-3k7b8Q-2}\\{NW4bhQ2K-3k7b8Q-3}\\{NW4bhQ2K-3k7b8Q-4}\\{NW4bhQ2K-3k7b8Q-5}\\{NW4bhQ2K-3k7b8Q-6}\\{NW4bhQ2K-3k7b8Q-7}\\{NW4bhQ2K-3k7b8Q-8}\\{NW4bhQ2K-3k7b8Q-9}}\nwnotused{show.c}\nwendcode{}\nwbegindocs{2}\nwdocspar
The external entrypoints \texttt{show} and \texttt{dval} just call the
main worker function \texttt{showp}. This function is called with
three additional parameters. The first is an integer flag denoting
whether a constructor application has to be enclosed in parentheses,
the second is the rest of the string, and the last is an integer flag
set to 1 if \texttt{showp} is called from \texttt{show} and 0
otherwise.

\nwenddocs{}\nwbegincode{3}\sublabel{NW4bhQ2K-3k7b8Q-2}\nwmargintag{{\nwtagstyle{}\subpageref{NW4bhQ2K-3k7b8Q-2}}}\moddef{show.c~{\nwtagstyle{}\subpageref{NW4bhQ2K-3k7b8Q-1}}}\plusendmoddef\nwstartdeflinemarkup\nwprevnextdefs{NW4bhQ2K-3k7b8Q-1}{NW4bhQ2K-3k7b8Q-3}\nwenddeflinemarkup
FUNCTION(__show)
\{
    EXPORT_LABEL(__show)
 ENTRY_LABEL(__show)
    CHECK_STACK(3);
    sp   -= 3;
    sp[0] = sp[3];
    sp[1] = zero;
    sp[2] = nil;
    sp[3] = one;
    GOTO(showp);
\}

FUNCTION(__dval)
\{
    EXPORT_LABEL(__dval)
 ENTRY_LABEL(__dval)
    CHECK_STACK(3);
    sp   -= 3;
    sp[0] = sp[3];
    sp[1] = zero;
    sp[2] = nil;
    sp[3] = zero;
    GOTO(showp);
\}

\nwendcode{}\nwbegindocs{4}\nwdocspar
When necessary, the \texttt{showp} function evaluates its argument to
weak head normal form and makes sure that it is not an unbound
variable. Then, it dispatches on the type of the node and delegates
processing to special purpose functions in order to handle data
structures with arguments.

\nwenddocs{}\nwbegincode{5}\sublabel{NW4bhQ2K-3k7b8Q-3}\nwmargintag{{\nwtagstyle{}\subpageref{NW4bhQ2K-3k7b8Q-3}}}\moddef{show.c~{\nwtagstyle{}\subpageref{NW4bhQ2K-3k7b8Q-1}}}\plusendmoddef\nwstartdeflinemarkup\nwprevnextdefs{NW4bhQ2K-3k7b8Q-2}{NW4bhQ2K-3k7b8Q-4}\nwenddeflinemarkup
static
FUNCTION(showp_eval)
\{
    Node *clos;
 ENTRY_LABEL(showp_eval)
    CHECK_STACK(3);
    clos  = sp[0];
    sp   -= 3;
    sp[0] = clos->cl.args[0];
    sp[1] = clos->cl.args[1];
    sp[2] = clos->cl.args[2];
    sp[3] = clos->cl.args[3];
    GOTO(showp);
\}

static
FUNCTION(showp_lazy)
\{
    Node *susp, *clos;
 ENTRY_LABEL(showp_lazy)
    susp = sp[0];

    /* suspend the search if the node is not local */
    if ( !is_local_space(susp->s.spc) )
        GOTO(suspend_thread(resume, susp));

    /* lock the suspension */
    clos = susp->s.fn;
    SAVE(susp, q.wq);
    susp->info = &queueMe_info;
    susp->q.wq = (ThreadQueue)0;

    /* create an update frame */
    CHECK_STACK(5);
    sp   -= 5;
    sp[0] = clos->cl.args[0];
    sp[1] = clos->cl.args[1];
    sp[2] = clos->cl.args[2];
    sp[3] = clos->cl.args[3];
    sp[4] = (Node *)update;

    /* enter the callee */
    GOTO(showp);
\}

static
FUNCTION(showp)
\{
 ENTRY_LABEL(showp)
    TRACE(("%I enter showp%V\\n", 4, sp));
    GOTO(showp_1);
\}

static
FUNCTION(showp_1)
\{
    boolean      isop;
    boolean      isneg;
    char         buf[32], *cp;
    const char   *str;
    unsigned int i, argc;
    double       d;
    Node         *node, *clos, *cons, *tail, *arglist, **argp, *prec, *show;

 ENTRY_LABEL(showp_1)

    str   = buf;
    argc  = 0;
    node  = sp[0];
    show  = sp[3];
    isop  = false;
    isneg = false;

 again:
#if !ONLY_BOXED_OBJECTS
    if ( is_unboxed(node) )
    \{
        sprintf(buf, "%ld", unboxed_val(node));
        isneg = buf[0] == '-';
    \}
    else
#endif
        switch ( node_tag(node) )
        \{
        case INDIR_TAG:
            node = node->n.node;
            goto again;

        case CLOSURE_TAG:
        case SUSPEND_TAG:
        case QUEUEME_TAG:
            if ( show != zero )
            \{
                CHECK_STACK1();
                sp   -= 1;
                sp[0] = node;
                sp[1] = (Node *)showp_1;
                GOTO(node->info->eval);
            \}
            str = "_";
            break;

        case VARIABLE_TAG:
            if ( show != zero )
            \{
                sp[0] = node;
                GOTO(delay_thread(showp_1, node));
            \}
            str = lookup_name(node);
            break;

        case CHAR_TAG:
            sprintf(buf, "'%s'", lit_char(node->ch.ch, '\\''));
            break;

#if ONLY_BOXED_OBJECTS
        case INT_TAG:
            sprintf(buf, "%ld", node->i.i);
            isneg = buf[0] == '-';
            break;
#endif

        case FLOAT_TAG:
            get_float_val(d, node->f);
            sprintf(buf, "%g", d);

            cp = strpbrk(buf, ".e");
            if ( cp == (char *)0 )
                strcat(buf, ".0");
            else if ( *cp != '.'  )
            \{
                i = strlen(cp) + 1;
                for ( cp += i; i-- > 0; cp-- )
                    cp[2] = cp[0];
                cp[1] = '.';
                cp[2] = '0';
            \}
            isneg = buf[0] == '-';
            break;

        case PAPP_TAG:
            argc = closure_argc(node);
            str  = node->info->cname;
            isop = is_operator(node->info);
            break;

        case SEARCH_CONT_TAG:
            str = node->info->cname;
            break;

        default:
            ASSERT(is_constr_node(node) || is_abstract_node(node));
            if ( is_abstract_node(node) )
            \{
                str = node->info->cname;
                if ( str == (const char *)0 )
                    str = "<abstract>";
            \}
            else if ( node->info == (NodeInfo *)&cons_info )
            \{
                sp[0] = node->c.args[0];
                sp[1] = node->c.args[1];
                GOTO(showList);
            \}
            else if ( is_tuple(node->info) )
            \{
                *++sp = node;
                GOTO(showTuple);
            \}
            else
            \{
                str  = node->info->cname;
                argc = is_vector(node) ? vector_argc(node)
                                       : constr_argc(node);
            \}
            isop = node->info->cname && is_operator(node->info);
        \}

    if ( isop && argc == 2 )
    \{
        sp[0] = node;
        GOTO(showInfix);
    \}

    sp[0] = node;
    CHECK_HEAP((argc + 1) * cons_node_size + closure_node_size(4)
               + suspend_node_size);
    node = sp[0];
    prec = sp[1];
    tail = sp[2];
    show = sp[3];

    if ( argc > 0 )
    \{
        if ( is_papp_node(node) )
            argp = node->cl.args;
        else if ( is_vector(node) )
            argp = node->a.args;
        else
            argp = node->c.args;

        arglist = nil;
        for ( i = argc; i-- > 0; )
        \{
            cons            = (Node *)hp;
            cons->c.info    = &cons_info;
            cons->c.args[0] = argp[i];
            cons->c.args[1] = arglist;
            arglist         = cons;
            hp             += cons_node_size;
        \}

        clos             = (Node *)hp;
        clos->cl.info    = &showArgs_info;
        clos->cl.args[0] = arglist;
        clos->cl.args[1] = prec;
        clos->cl.args[2] = tail;
        clos->cl.args[3] = show;
        hp              += closure_node_size(4);

        tail         = (Node *)hp;
        tail->s.info = &showArgs_suspend_info;
        tail->s.fn   = clos;
        tail->s.spc  = ss;
        hp          += suspend_node_size;
    \}

    if ( isop || (isneg && int_val(prec) > 0) )
    \{
        cons            = (Node *)hp;
        cons->c.info    = &cons_info;
        cons->c.args[0] = rparen;
        cons->c.args[1] = tail;
        hp             += cons_node_size;

        tail = cons;
    \}

    sp[0] = prefix_string(str, tail);

    CHECK_HEAP(2*cons_node_size);
    cons = sp[0];
    prec = sp[1];
    sp  += 4;
    if ( isop || (isneg && int_val(prec) > 0) )
    \{
        tail            = cons;
        cons            = (Node *)hp;
        cons->c.info    = &cons_info;
        cons->c.args[0] = lparen;
        cons->c.args[1] = tail;
        hp             += cons_node_size;
    \}
    if ( int_val(prec) > 1 && argc > 0 )
    \{
        tail            = cons;
        cons            = (Node *)hp;
        cons->c.info    = &cons_info;
        cons->c.args[0] = lparen;
        cons->c.args[1] = tail;
        hp             += cons_node_size;
    \}

    RETURN(cons);
\}

\nwendcode{}\nwbegindocs{6}\nwdocspar
The function \texttt{showArgs} displays the arguments of a constructor
application. Each argument is preceded by a space and enclosed in
parentheses if necessary.

\nwenddocs{}\nwbegincode{7}\sublabel{NW4bhQ2K-3k7b8Q-4}\nwmargintag{{\nwtagstyle{}\subpageref{NW4bhQ2K-3k7b8Q-4}}}\moddef{show.c~{\nwtagstyle{}\subpageref{NW4bhQ2K-3k7b8Q-1}}}\plusendmoddef\nwstartdeflinemarkup\nwprevnextdefs{NW4bhQ2K-3k7b8Q-3}{NW4bhQ2K-3k7b8Q-5}\nwenddeflinemarkup
static
FUNCTION(showArgs_eval)
\{
    Node *clos;
 ENTRY_LABEL(showArgs_eval)
    CHECK_STACK(3);
    clos  = sp[0];
    sp   -= 3;
    sp[0] = clos->cl.args[0];
    sp[1] = clos->cl.args[1];
    sp[2] = clos->cl.args[2];
    sp[3] = clos->cl.args[3];
    GOTO(showArgs);
\}

static
FUNCTION(showArgs_lazy)
\{
    Node *susp, *clos;
 ENTRY_LABEL(showArgs_lazy)
    susp = sp[0];

    /* suspend the search if the node is not local */
    if ( !is_local_space(susp->s.spc) )
        GOTO(suspend_thread(resume, susp));

    /* lock the suspension */
    clos = susp->s.fn;
    SAVE(susp, q.wq);
    susp->info = &queueMe_info;
    susp->q.wq = (ThreadQueue)0;

    /* create an update frame */
    CHECK_STACK(5);
    sp   -= 5;
    sp[0] = clos->cl.args[0];
    sp[1] = clos->cl.args[1];
    sp[2] = clos->cl.args[2];
    sp[3] = clos->cl.args[3];
    sp[4] = (Node *)update;

    /* enter the callee */
    GOTO(showArgs);
\}

static
FUNCTION(showArgs)
\{
    Node *arglist, *clos, *susp, *cons, *tail, *prec, *show;

 ENTRY_LABEL(showArgs)
    TRACE(("%I enter showArgs%V\\n", 4, sp));
    CHECK_HEAP(2 * (closure_node_size(4) + suspend_node_size)
               + cons_node_size);
    arglist = sp[0];
    prec    = sp[1];
    tail    = sp[2];
    show    = sp[3];
    sp     += 4;

    if ( arglist->info->tag == NIL_TAG )
    \{
        if ( int_val(prec) > 1 )
        \{
            cons            = (Node *)hp;
            cons->c.info    = &cons_info;
            cons->c.args[0] = rparen;
            cons->c.args[1] = tail;
            hp             += cons_node_size;
        \}
        else
        \{
            *--sp = tail;
            GOTO(tail->info->eval);
        \}
    \}
    else
    \{
        ASSERT(arglist->info->tag == CONS_TAG);

        clos             = (Node *)hp;
        clos->cl.info    = &showArgs_info;
        clos->cl.args[0] = arglist->c.args[1];
        clos->cl.args[1] = prec;
        clos->cl.args[2] = tail;
        clos->cl.args[3] = show;
        hp              += closure_node_size(4);

        susp         = (Node *)hp;
        susp->s.info = &showArgs_suspend_info;
        susp->s.fn   = clos;
        susp->s.spc  = ss;
        hp          += suspend_node_size;

        clos             = (Node *)hp;
        clos->cl.info    = &showp_info;
        clos->cl.args[0] = arglist->c.args[0];
        clos->cl.args[1] = two;
        clos->cl.args[2] = susp;
        clos->cl.args[3] = show;
        hp              += closure_node_size(4);

        susp         = (Node *)hp;
        susp->s.info = &showp_suspend_info;
        susp->s.fn   = clos;
        susp->s.spc  = ss;
        hp          += suspend_node_size;

        cons            = (Node *)hp;
        cons->c.info    = &cons_info;
        cons->c.args[0] = space;
        cons->c.args[1] = susp;
        hp             += cons_node_size;
    \}

    RETURN(cons);
\}

\nwendcode{}\nwbegindocs{8}\nwdocspar
The function \texttt{showInfix} is invoked when the node to be shown
is an application of an infix operator to exactly two arguments. Such
nodes are displayed in infix notation. As we do not have access to the
fixity of the operator, we will ensure that every nested infix
application is enclosed in parentheses.

\nwenddocs{}\nwbegincode{9}\sublabel{NW4bhQ2K-3k7b8Q-5}\nwmargintag{{\nwtagstyle{}\subpageref{NW4bhQ2K-3k7b8Q-5}}}\moddef{show.c~{\nwtagstyle{}\subpageref{NW4bhQ2K-3k7b8Q-1}}}\plusendmoddef\nwstartdeflinemarkup\nwprevnextdefs{NW4bhQ2K-3k7b8Q-4}{NW4bhQ2K-3k7b8Q-6}\nwenddeflinemarkup
static
FUNCTION(showInfix)
\{
    int prec;
    Node *node, *tail, *show, *cons, *clos, *susp;
 ENTRY_LABEL(showInfix)

    CHECK_HEAP(2 * cons_node_size + closure_node_size(4) + suspend_node_size);
    node = sp[0];
    prec = int_val(sp[1]);
    tail = sp[2];
    show = sp[3];

    if ( prec > 0 )
    \{
        cons            = (Node *)hp;
        cons->c.info    = &cons_info;
        cons->c.args[0] = rparen;
        cons->c.args[1] = tail;
        hp             += cons_node_size;

        tail = cons;
    \}

    clos             = (Node *)hp;
    clos->cl.info    = &showp_info;
    clos->cl.args[0] = node->c.args[1];
    clos->cl.args[1] = one;
    clos->cl.args[2] = tail;
    clos->cl.args[3] = show;
    hp              += closure_node_size(4);

    susp         = (Node *)hp;
    susp->s.info = &showp_suspend_info;
    susp->s.fn   = clos;
    susp->s.spc  = ss;
    hp          += suspend_node_size;

    cons            = (Node *)hp;
    cons->c.info    = &cons_info;
    cons->c.args[0] = space;
    cons->c.args[1] = susp;
    hp             += cons_node_size;

    sp[0] = node->c.args[0];
    sp[2] = prefix_string(node->info->cname, cons);

    CHECK_HEAP(2 * cons_node_size + closure_node_size(4) + suspend_node_size);
    node = sp[0];
    tail = sp[2];
    show = sp[3];
    sp  += 4;

    cons            = (Node *)hp;
    cons->c.info    = &cons_info;
    cons->c.args[0] = space;
    cons->c.args[1] = tail;
    hp             += cons_node_size;

    clos             = (Node *)hp;
    clos->cl.info    = &showp_info;
    clos->cl.args[0] = node;
    clos->cl.args[1] = one;
    clos->cl.args[2] = cons;
    clos->cl.args[3] = show;
    hp              += closure_node_size(4);

    susp         = (Node *)hp;
    susp->s.info = &showp_suspend_info;
    susp->s.fn   = clos;
    susp->s.spc  = ss;
    hp          += suspend_node_size;

    if ( prec > 0 )
    \{
        cons            = (Node *)hp;
        cons->c.info    = &cons_info;
        cons->c.args[0] = lparen;
        cons->c.args[1] = susp;
        hp             += cons_node_size;
    \}
    else
        cons = susp;

    RETURN(cons);
\}

\nwendcode{}\nwbegindocs{10}\nwdocspar
The code of \texttt{showp} jumps to the label \texttt{showTuple} in
order display a tuple. In this case, the arguments are separated by
commas and not enclosed in parentheses. Instead, the whole tuple is
always enclosed in parentheses. Note that the function is prepared to
handle tuples as well as vector-like data which is shown like a
tuple.

\nwenddocs{}\nwbegincode{11}\sublabel{NW4bhQ2K-3k7b8Q-6}\nwmargintag{{\nwtagstyle{}\subpageref{NW4bhQ2K-3k7b8Q-6}}}\moddef{show.c~{\nwtagstyle{}\subpageref{NW4bhQ2K-3k7b8Q-1}}}\plusendmoddef\nwstartdeflinemarkup\nwprevnextdefs{NW4bhQ2K-3k7b8Q-5}{NW4bhQ2K-3k7b8Q-7}\nwenddeflinemarkup
static
FUNCTION(showTuple)
\{
    unsigned int i, argc;
    Node         *tuple, *clos, *susp, *cons, *tail, *arglist, **argp, *show;

 ENTRY_LABEL(showTuple)

    argc = is_vector(sp[0]) ? vector_argc(sp[0]) : constr_argc(sp[0]);
    CHECK_HEAP(argc * cons_node_size + closure_node_size(3)
               + closure_node_size(4) + 2*suspend_node_size);
    tuple = sp[0];
    tail  = sp[1];
    show  = sp[2];
    sp   += 3;

    argp    = is_vector(tuple) ? tuple->a.args : tuple->c.args;
    argp   += argc;
    arglist = nil;
    for ( i = argc; i-- > 1; )
    \{
        cons            = (Node *)hp;
        cons->c.info    = &cons_info;
        cons->c.args[0] = *--argp;
        cons->c.args[1] = arglist;
        arglist         = cons;
        hp             += cons_node_size;
    \}

    clos             = (Node *)hp;
    clos->cl.info    = &showTupleArgs_info;
    clos->cl.args[0] = arglist;
    clos->cl.args[1] = tail;
    clos->cl.args[2] = show;
    hp              += closure_node_size(3);

    susp         = (Node *)hp;
    susp->s.info = &showTupleArgs_suspend_info;
    susp->s.fn   = clos;
    susp->s.spc  = ss;
    hp          += suspend_node_size;

    clos             = (Node *)hp;
    clos->cl.info    = &showp_info;
    clos->cl.args[0] = *--argp;
    clos->cl.args[1] = zero;
    clos->cl.args[2] = susp;
    clos->cl.args[3] = show;
    hp              += closure_node_size(4);

    susp         = (Node *)hp;
    susp->s.info = &showp_suspend_info;
    susp->s.fn   = clos;
    susp->s.spc  = ss;
    hp          += suspend_node_size;

    cons            = (Node *)hp;
    cons->c.info    = &cons_info;
    cons->c.args[0] = lparen;
    cons->c.args[1] = susp;
    hp             += cons_node_size;

    RETURN(cons);
\}

static
FUNCTION(showTupleArgs_eval)
\{
    Node *clos;
 ENTRY_LABEL(showTupleArgs_eval)
    CHECK_STACK(2);
    clos  = sp[0];
    sp   -= 2;
    sp[0] = clos->cl.args[0];
    sp[1] = clos->cl.args[1];
    sp[2] = clos->cl.args[2];
    GOTO(showTupleArgs);
\}

static
FUNCTION(showTupleArgs_lazy)
\{
    Node *susp, *clos;
 ENTRY_LABEL(showTupleArgs_lazy)
    susp = sp[0];

    /* suspend the search if the node is not local */
    if ( !is_local_space(susp->s.spc) )
        GOTO(suspend_thread(resume, susp));

    /* lock the suspension */
    clos = susp->s.fn;
    SAVE(susp, q.wq);
    susp->info = &queueMe_info;
    susp->q.wq = (ThreadQueue)0;

    /* create an update frame */
    CHECK_STACK(4);
    sp   -= 4;
    sp[0] = clos->cl.args[0];
    sp[1] = clos->cl.args[1];
    sp[2] = clos->cl.args[2];
    sp[3] = (Node *)update;

    /* enter the callee */
    GOTO(showTupleArgs);
\}

static
FUNCTION(showTupleArgs)
\{
    Node *arglist, *clos, *susp, *cons, *tail, *show;

 ENTRY_LABEL(showTupleArgs)
    TRACE(("%I enter showTupleArgs%V\\n", 3, sp));
    CHECK_HEAP(closure_node_size(3) + closure_node_size(4)
               + 2*suspend_node_size + cons_node_size);
    arglist = sp[0];
    tail    = sp[1];
    show    = sp[2];
    sp     += 3;

    if ( arglist->info->tag == NIL_TAG )
    \{
        cons            = (Node *)hp;
        cons->c.info    = &cons_info;
        cons->c.args[0] = rparen;
        cons->c.args[1] = tail;
        hp             += cons_node_size;
    \}
    else
    \{
        ASSERT(arglist->info->tag == CONS_TAG);

        clos             = (Node *)hp;
        clos->cl.info    = &showTupleArgs_info;
        clos->cl.args[0] = arglist->c.args[1];
        clos->cl.args[1] = tail;
        clos->cl.args[2] = show;
        hp              += closure_node_size(3);

        susp         = (Node *)hp;
        susp->s.info = &showTupleArgs_suspend_info;
        susp->s.fn   = clos;
        susp->s.spc  = ss;
        hp          += suspend_node_size;

        clos             = (Node *)hp;
        clos->cl.info    = &showp_info;
        clos->cl.args[0] = arglist->c.args[0];
        clos->cl.args[1] = zero;
        clos->cl.args[2] = susp;
        clos->cl.args[3] = show;
        hp              += closure_node_size(4);

        susp         = (Node *)hp;
        susp->s.info = &showp_suspend_info;
        susp->s.fn   = clos;
        susp->s.spc  = ss;
        hp          += suspend_node_size;

        cons            = (Node *)hp;
        cons->c.info    = &cons_info;
        cons->c.args[0] = comma;
        cons->c.args[1] = susp;
        hp             += cons_node_size;
    \}

    RETURN(cons);
\}

\nwendcode{}\nwbegindocs{12}\nwdocspar
Lists are printed in the standard list notation, enclosing the whole
list in brackets and separating the arguments by commas. Eventually
the function \texttt{showTail} has to evaluate its tail before it can
be converted into a string. When being called from \texttt{dval}, we do
not evaluate the tail of the list, but print it in the usual Prolog
notation, i.e., separating the unevaluated or non-ground tail from the
head of the list with a vertical bar. Unfortunately, this notation is
not available in the source code because of ambiguities with the list
comprehension syntax. Fortunately, list comprehensions cannot occur in
the output; thus, there is no ambiguity here.

When we print a character list whose elements are ground values, we
make use of the standard string notation. In the case of the
\texttt{show} function we only need to check the first element of the
list in order to check whether it is a character. As \texttt{show} is
rigid we know that the string will be ground. The situation is
different for \texttt{dval} which is not rigid. Therefore, we would
have to check every character of the string. As the argument to
\texttt{dval} might be an infinite list\footnote{Such lists can be
constructed with recursive local bindings, e.g., \texttt{repeat x = xs
where xs = x:xs}. Note that this is an extension to the Curry language
that is supported by the compiler.}, we will never the use string
notation for \texttt{dval}.

\nwenddocs{}\nwbegincode{13}\sublabel{NW4bhQ2K-3k7b8Q-7}\nwmargintag{{\nwtagstyle{}\subpageref{NW4bhQ2K-3k7b8Q-7}}}\moddef{show.c~{\nwtagstyle{}\subpageref{NW4bhQ2K-3k7b8Q-1}}}\plusendmoddef\nwstartdeflinemarkup\nwprevnextdefs{NW4bhQ2K-3k7b8Q-6}{NW4bhQ2K-3k7b8Q-8}\nwenddeflinemarkup
static
FUNCTION(showList)
\{
    Node *hd, *tl, *clos, *susp, *cons, *tail, *show;

 ENTRY_LABEL(showList)
    hd = sp[0];
    show = sp[3];
    if ( show != zero )
    \{
    again:
        if ( is_boxed(hd) )
            switch ( hd->info->tag )
            \{
            case INDIR_TAG:
                hd = hd->n.node;
                goto again;
            case SUSPEND_TAG:
            case QUEUEME_TAG:
            case CLOSURE_TAG:
                CHECK_STACK1();
                sp -= 1;
                sp[0] = hd;
                sp[1] = (Node *)showList;
                GOTO(hd->info->eval);
            case VARIABLE_TAG:
                sp[0] = hd;
                GOTO(delay_thread(showList, hd));
            case CHAR_TAG:
                sp[0] = hd;
                CHECK_HEAP(closure_node_size(2) + suspend_node_size
                           + 2*cons_node_size);

                hd   = sp[0];
                tl   = sp[1];
                tail = sp[2];
                sp  += 4;

                cons = (Node *)hp;
                cons->c.info = &cons_info;
                cons->c.args[0] = hd;
                cons->c.args[1] = tl;
                hp             += cons_node_size;

                clos             = (Node *)hp;
                clos->cl.info    = &showStringTail_info;
                clos->cl.args[0] = cons;
                clos->cl.args[1] = tail;
                hp              += closure_node_size(2);

                susp         = (Node *)hp;
                susp->s.info = &showStringTail_suspend_info;
                susp->s.fn   = clos;
                susp->s.spc  = ss;
                hp          += suspend_node_size;

                cons            = (Node *)hp;
                cons->c.info    = &cons_info;
                cons->c.args[0] = dblquote;
                cons->c.args[1] = susp;
                hp             += cons_node_size;

                RETURN(cons);
            \}
        sp[0] = hd;
    \}

    CHECK_HEAP(closure_node_size(3) + closure_node_size(4)
               + 2*suspend_node_size + cons_node_size);
    hd   = sp[0];
    tl   = sp[1];
    tail = sp[2];
    show = sp[3];
    sp  += 4;

    clos             = (Node *)hp;
    clos->cl.info    = &showTail_info;
    clos->cl.args[0] = tl;
    clos->cl.args[1] = tail;
    clos->cl.args[2] = show;
    hp              += closure_node_size(3);

    susp         = (Node *)hp;
    susp->s.info = &showTail_suspend_info;
    susp->s.fn   = clos;
    susp->s.spc  = ss;
    hp          += suspend_node_size;

    clos             = (Node *)hp;
    clos->cl.info    = &showp_info;
    clos->cl.args[0] = hd;
    clos->cl.args[1] = zero;
    clos->cl.args[2] = susp;
    clos->cl.args[3] = show;
    hp              += closure_node_size(4);

    susp         = (Node *)hp;
    susp->s.info = &showp_suspend_info;
    susp->s.fn   = clos;
    susp->s.spc  = ss;
    hp          += suspend_node_size;

    cons            = (Node *)hp;
    cons->c.info    = &cons_info;
    cons->c.args[0] = lbracket;
    cons->c.args[1] = susp;
    hp             += cons_node_size;

    RETURN(cons);
\}

static
FUNCTION(showTail_eval)
\{
    Node *clos;
 ENTRY_LABEL(showTail_eval)
    CHECK_STACK(2);
    clos  = sp[0];
    sp   -= 2;
    sp[0] = clos->cl.args[0];
    sp[1] = clos->cl.args[1];
    sp[2] = clos->cl.args[2];
    GOTO(showTail);
\}

static
FUNCTION(showTail_lazy)
\{
    Node *susp, *clos;
 ENTRY_LABEL(showTail_lazy)
    susp = sp[0];

    /* suspend the search if the node is not local */
    if ( !is_local_space(susp->s.spc) )
        GOTO(suspend_thread(resume, susp));

    /* lock the suspension */
    clos = susp->s.fn;
    SAVE(susp, q.wq);
    susp->info = &queueMe_info;
    susp->q.wq = (ThreadQueue)0;

    /* create an update frame */
    CHECK_STACK(4);
    sp   -= 4;
    sp[0] = clos->cl.args[0];
    sp[1] = clos->cl.args[1];
    sp[2] = clos->cl.args[2];
    sp[3] = (Node *)update;

    /* enter the callee */
    GOTO(showTail);
\}

static
FUNCTION(showTail)
\{
 ENTRY_LABEL(showTail)
    TRACE(("%I enter showTail%V\\n", 3, sp));
    GOTO(showTail_1);
\}

static
FUNCTION(showTail_1)
\{
    Node *list, *clos, *susp, *cons, *tail, *show;

 ENTRY_LABEL(showTail_1)
    list = sp[0];
    show = sp[2];
 again:
    switch ( node_tag(list) )
    \{
    case INDIR_TAG:
        list = list->n.node;
        goto again;

    case CLOSURE_TAG:
    case SUSPEND_TAG:
    case QUEUEME_TAG:
        if ( show != zero )
        \{
            CHECK_STACK1();
            sp -= 1;
            sp[0] = list;
            sp[1] = (Node *)showTail_1;
            GOTO(list->info->eval);
        \}
        goto make_tail;

    case VARIABLE_TAG:
        if ( show != zero )
        \{
            sp[0] = list;
            GOTO(delay_thread(showTail_1, list));
        \}
    make_tail:
        sp[0] = list;
        CHECK_HEAP(closure_node_size(4) + suspend_node_size
                   + 2*cons_node_size);
        list = sp[0];
        tail = sp[1];
        show = sp[2];
        sp  += 3;

        cons            = (Node *)hp;
        cons->c.info    = &cons_info;
        cons->c.args[0] = rbracket;
        cons->c.args[1] = tail;
        hp             += cons_node_size;

        clos             = (Node *)hp;
        clos->cl.info    = &showp_info;
        clos->cl.args[0] = list;
        clos->cl.args[1] = zero;
        clos->cl.args[2] = cons;
        clos->cl.args[3] = show;
        hp              += closure_node_size(4);

        susp         = (Node *)hp;
        susp->s.info = &showp_suspend_info;
        susp->s.fn   = clos;
        susp->s.spc  = ss;
        hp          += suspend_node_size;

        cons            = (Node *)hp;
        cons->c.info    = &cons_info;
        cons->c.args[0] = bar;
        cons->c.args[1] = susp;
        hp             += cons_node_size;
        break;

    case NIL_TAG:
        CHECK_HEAP(cons_node_size);
        tail = sp[1];
        sp  += 3;

        cons            = (Node *)hp;
        cons->c.info    = &cons_info;
        cons->c.args[0] = rbracket;
        cons->c.args[1] = tail;
        hp             += cons_node_size;
        break;

    case CONS_TAG:
        sp[0] = list;
        CHECK_HEAP(closure_node_size(3) + closure_node_size(4)
                   + 2*suspend_node_size + cons_node_size);
        list = sp[0];
        tail = sp[1];
        show = sp[2];
        sp  += 3;

        clos             = (Node *)hp;
        clos->cl.info    = &showTail_info;
        clos->cl.args[0] = list->c.args[1];
        clos->cl.args[1] = tail;
        clos->cl.args[2] = show;
        hp              += closure_node_size(3);

        susp         = (Node *)hp;
        susp->s.info = &showTail_suspend_info;
        susp->s.fn   = clos;
        susp->s.spc  = ss;
        hp          += suspend_node_size;

        clos             = (Node *)hp;
        clos->cl.info    = &showp_info;
        clos->cl.args[0] = list->c.args[0];
        clos->cl.args[1] = zero;
        clos->cl.args[2] = susp;
        clos->cl.args[3] = show;
        hp              += closure_node_size(4);

        susp         = (Node *)hp;
        susp->s.info = &showp_suspend_info;
        susp->s.fn   = clos;
        susp->s.spc  = ss;
        hp          += suspend_node_size;

        cons            = (Node *)hp;
        cons->c.info    = &cons_info;
        cons->c.args[0] = comma;
        cons->c.args[1] = susp;
        hp             += cons_node_size;
        break;

    default:
        fprintf(stderr, "Bad list tail in showTail\\n");
        exit(1);
    \}

    RETURN(cons);
\}

\nwendcode{}\nwbegindocs{14}\nwdocspar
The \texttt{showString} code is used to print a non-empty string head.
It evaluates the head of the string to a character node and prints
that character. It will then continue to display the remaining string.

\nwenddocs{}\nwbegincode{15}\sublabel{NW4bhQ2K-3k7b8Q-8}\nwmargintag{{\nwtagstyle{}\subpageref{NW4bhQ2K-3k7b8Q-8}}}\moddef{show.c~{\nwtagstyle{}\subpageref{NW4bhQ2K-3k7b8Q-1}}}\plusendmoddef\nwstartdeflinemarkup\nwprevnextdefs{NW4bhQ2K-3k7b8Q-7}{NW4bhQ2K-3k7b8Q-9}\nwenddeflinemarkup
static
FUNCTION(showString)
\{
    Node *hd, *tl, *tail, *clos, *susp, *cons;

 ENTRY_LABEL(showString);
    EVAL_RIGID_CHAR(showString);
    CHECK_HEAP(closure_node_size(2) + suspend_node_size);

    hd   = sp[0];
    tl   = sp[1];
    tail = sp[2];
    sp  += 3;

    clos             = (Node *)hp;
    clos->cl.info    = &showStringTail_info;
    clos->cl.args[0] = tl;
    clos->cl.args[1] = tail;
    hp              += closure_node_size(2);

    susp         = (Node *)hp;
    susp->s.info = &showStringTail_suspend_info;
    susp->s.fn   = clos;
    susp->s.spc  = ss;
    hp          += suspend_node_size;

    cons = prefix_string(lit_char(hd->ch.ch, '"'), susp);
    RETURN(cons);
\}

static
FUNCTION(showStringTail_eval)
\{
    Node *clos;
 ENTRY_LABEL(showStringTail_eval)
    CHECK_STACK(1);
    clos  = sp[0];
    sp   -= 1;
    sp[0] = clos->cl.args[0];
    sp[1] = clos->cl.args[1];
    GOTO(showStringTail);
\}

static
FUNCTION(showStringTail_lazy)
\{
    Node *susp, *clos;
 ENTRY_LABEL(showStringTail_lazy)
    susp = sp[0];

    /* suspend the search if the node is not local */
    if ( !is_local_space(susp->s.spc) )
        GOTO(suspend_thread(resume, susp));

    /* lock the suspension */
    clos = susp->s.fn;
    SAVE(susp, q.wq);
    susp->info = &queueMe_info;
    susp->q.wq = (ThreadQueue)0;

    /* create an update frame */
    CHECK_STACK(3);
    sp   -= 3;
    sp[0] = clos->cl.args[0];
    sp[1] = clos->cl.args[1];
    sp[2] = (Node *)update;

    /* enter the callee */
    GOTO(showStringTail);
\}

static
FUNCTION(showStringTail)
\{
    Node *cons;

 ENTRY_LABEL(showStringTail)
    EVAL_RIGID_POLY(showStringTail);

    if ( sp[0] == nil )
    \{
        CHECK_HEAP(cons_node_size);

        cons            = (Node *)hp;
        cons->c.info    = &cons_info;
        cons->c.args[0] = dblquote;
        cons->c.args[1] = sp[1];
        hp             += cons_node_size;

        sp += 2;
        RETURN(cons);
    \}

    ASSERT(sp[0]->info == &cons_info);
    CHECK_STACK1();
    sp   -= 1;
    sp[0] = sp[1]->c.args[0];
    sp[1] = sp[1]->c.args[1];
    GOTO(showString);
\}

\nwendcode{}\nwbegindocs{16}\nwdocspar
The functions \verb|showEFloat| and \verb|showFFloat| convert a
floating-point number into a string using scientific and fixed
point formats, respectively. In order to avoid buffer overflows, these
functions use a dynamically allocated temporary buffer. If the
precision argument is not negative, it specifies the number of decimal
digits for the result.

\nwenddocs{}\nwbegincode{17}\sublabel{NW4bhQ2K-3k7b8Q-9}\nwmargintag{{\nwtagstyle{}\subpageref{NW4bhQ2K-3k7b8Q-9}}}\moddef{show.c~{\nwtagstyle{}\subpageref{NW4bhQ2K-3k7b8Q-1}}}\plusendmoddef\nwstartdeflinemarkup\nwprevnextdefs{NW4bhQ2K-3k7b8Q-8}{\relax}\nwenddeflinemarkup
DECLARE_ENTRYPOINT(__showEFloat);
DECLARE_ENTRYPOINT(__showFFloat);

DECLARE_LABEL(__showEFloat_1);
DECLARE_LABEL(__showFFloat_1);

FUNCTION(__showEFloat)
\{
    Node *p, *d;
    EXPORT_LABEL(__showEFloat)
 ENTRY_LABEL(__showEFloat)
    EVAL_RIGID_INT(__showEFloat);
    p     = sp[0];
    d     = sp[1];
    sp[0] = d;
    sp[1] = p;
    GOTO(__showEFloat_1);
\}

static
FUNCTION(__showEFloat_1)
\{
    int    p, n;
    double d;
    char   fmt[20], *buf;
    Node   *str;
 ENTRY_LABEL(__showEFloat_1)
    EVAL_RIGID_FLOAT(__showEFloat_1);
    get_float_val(d, sp[0]->f);
    p   = int_val(sp[1]);
    str = sp[2];
    if ( p >= 0 )
        sprintf(fmt, "%%.%de", p);
    else
        strcpy(fmt, "%e");
    n   = p >= 0 ? 10 + p : 25;
    buf = (char *)malloc(n);
    if ( buf == (char *)0 )
    \{
        fprintf(stderr, "showEFloat: memory exhausted\\n");
        exit(1);
    \}
    sprintf(buf, fmt, d);

    sp += 3;
    str = prefix_string(buf, str);
    free(buf);
    RETURN(str);
\}

FUNCTION(__showFFloat)
\{
    Node *p, *d;
    EXPORT_LABEL(__showFFloat)
 ENTRY_LABEL(__showFFloat)
    EVAL_RIGID_INT(__showFFloat);
    p     = sp[0];
    d     = sp[1];
    sp[0] = d;
    sp[1] = p;
    GOTO(__showFFloat_1);
\}

static
FUNCTION(__showFFloat_1)
\{
    int    p, n;
    double d;
    char   fmt[20], *buf;
    Node   *str;
 ENTRY_LABEL(__showFFloat_1)
    EVAL_RIGID_FLOAT(__showFFloat_1);
    get_float_val(d, sp[0]->f);
    p   = int_val(sp[1]);
    str = sp[2];
    if ( p >= 0 )
        sprintf(fmt, "%%.%df", p);
    else
        strcpy(fmt, "%f");
    frexp(d, &n);
    if ( p >= 0 )
        n = (n > 0 ? n / 3 + 4 : 5) + p;
    else
        n = (n >= 0 ? n : -n) / 3 + 20;
    buf = (char *)malloc(n);
    if ( buf == (char *)0 )
    \{
        fprintf(stderr, "showFFloat: memory exhausted\\n");
        exit(1);
    \}
    sprintf(buf, fmt, d);

    sp += 3;
    str = prefix_string(buf, str);
    free(buf);
    RETURN(str);
\}
\nwendcode{}

\nwixlogsorted{c}{{show.c}{NW4bhQ2K-3k7b8Q-1}{\nwixd{NW4bhQ2K-3k7b8Q-1}\nwixd{NW4bhQ2K-3k7b8Q-2}\nwixd{NW4bhQ2K-3k7b8Q-3}\nwixd{NW4bhQ2K-3k7b8Q-4}\nwixd{NW4bhQ2K-3k7b8Q-5}\nwixd{NW4bhQ2K-3k7b8Q-6}\nwixd{NW4bhQ2K-3k7b8Q-7}\nwixd{NW4bhQ2K-3k7b8Q-8}\nwixd{NW4bhQ2K-3k7b8Q-9}}}%

