\nwfilename{curry.nw}\nwbegindocs{0}% -*- noweb-code-mode: c-mode -*-% ===> this file was generated automatically by noweave --- better not edit it
% $Id: curry.nw,v 2.3 2003/01/26 18:37:38 lux Exp $
%
% Copyright (c) 2002, Wolfgang Lux
% See ../LICENSE for the full license.
%
\section{The runtime system}
This chapter describes the C runtime system for the Curry abstract
machine. Clients can use the header file {\Tt{}curry.h\nwendquote} to include all
definitions for the runtime system. The file, in turn, includes the
header files of all runtime system components and also some required
header files from the standard C library.

\nwenddocs{}\nwbegincode{1}\sublabel{NW3YYKtn-2NNFqs-1}\nwmargintag{{\nwtagstyle{}\subpageref{NW3YYKtn-2NNFqs-1}}}\moddef{curry.h~{\nwtagstyle{}\subpageref{NW3YYKtn-2NNFqs-1}}}\endmoddef\nwstartdeflinemarkup\nwenddeflinemarkup
#ifndef _CURRY_H
#define _CURRY_H 1

/* configuration and debugging */
#include "config.h"
#include "debug.h"

/* standard C library */
#include <stdio.h>
#include <stdlib.h>
#include <string.h>
#include <math.h>

/* runtime system */
#include "run.h"
#include "heap.h"
#include "stack.h"
#include "trail.h"
#include "threads.h"
#include "spaces.h"
#include "eval.h"
#include "backtrack.h"
#include "io_monad.h"
#include "unify.h"
#include "cstring.h"
#include "main.h"
#include "trace.h"

/* abstract machine instructions */
#include "cam.h"

#endif /* _CURRY_H */

\nwnotused{curry.h}\nwendcode{}\nwbegindocs{2}\nwdocspar
\subsection{Configuration}
There are a few configuration options which control the runtime
system. The settings for these options are maintained in the file
{\Tt{}config.h\nwendquote} which in turn is generated from the template found in
{\Tt{}config.h.in\nwendquote} by the {\Tt{}configure\nwendquote} script.

Currently the following configuration options are supported.
\begin{itemize}
\item {\Tt{}UNALIGNED{\_}DOUBLE\nwendquote}: If this flag is set, double values can be
  accessed on a word boundary and need not be aligned on 8 byte
  boundaries. The value of this flag is determined automatically in
  the {\Tt{}configure\nwendquote} script.
\item {\Tt{}USE{\_}TRAMPOLINE\nwendquote}: This flag determines whether tail calls are
  implemented using a trampoline function or using direct jumps (see
  section~\ref{sec:execution}). This flag is controlled by the
  \texttt{--enable-trampoline} option of \texttt{configure}.
\item {\Tt{}ONLY{\_}BOXED{\_}OBJECTS\nwendquote}: This flag determines whether integer
  numbers are implemented as (boxed) nodes in the heap or by an
  unboxed encoding in the pointer. This flag is controlled by
  the \texttt{--disable-unboxed} option of \texttt{configure}.
\item {\Tt{}YIELD{\_}NONDET\nwendquote}: If this flag is non-zero, the abstract machine
  prioritizes deterministic threads over non-deterministic threads. By
  enabling this flag unnecessary non-deterministic splittings may be
  avoided, e.g. for the goal \verb|nat n & n =:= S Z| and the program
\begin{verbatim}
data Nat = Z | S Nat
nat Z     = success
nat (S n) = nat n
\end{verbatim}
  On the other hand, setting this flag may result in a high degree of
  concurrency and even lead to non-termination, e.g. for the goal
  \verb|repeat coin =:= from 0| and the program
\begin{verbatim}
coin = 0
coin = 1
repeat x = x : repeat x
from   n = n : from (n + 1)
\end{verbatim}
  Note that this non-termination is due to the implicit concurrent
  unification of the arguments of the two data structures. The setting
  of this flag is controlled by the \texttt{--enable-stability} option
  of \texttt{configure}.
\item {\Tt{}NO{\_}OCCURS{\_}CHECK\nwendquote}: If this flag is non-zero, the abstract
  machine does not perform an occurs check in the unification. This
  is slightly more efficient but leads to non-termination in examples
  like \verb|let x,y free in x =:= (1:y) & y =:= (1:x)|.
\end{itemize}
Note that the symbols {\Tt{}USE{\_}TRAMPOLINE\nwendquote}, {\Tt{}ONLY{\_}BOXED{\_}OBJECTS\nwendquote},
{\Tt{}YIELD{\_}NONDET\nwendquote}, and {\Tt{}NO{\_}OCCURS{\_}CHECK\nwendquote} are always defined. A
redefinition of these symbols therefore will raise a compilation error
thereby enforcing the consistent use of these options in the runtime
system.

\nwenddocs{}\nwbegincode{3}\sublabel{NW3YYKtn-2GRKeP-1}\nwmargintag{{\nwtagstyle{}\subpageref{NW3YYKtn-2GRKeP-1}}}\moddef{config.h.in~{\nwtagstyle{}\subpageref{NW3YYKtn-2GRKeP-1}}}\endmoddef\nwstartdeflinemarkup\nwenddeflinemarkup
#ifndef _CONFIG_H

/* Define to empty if doubles have alignment constraints */
#undef UNALIGNED_DOUBLE

/* Define as 1 to use trampoline code for tail calls */
#define USE_TRAMPOLINE 0

/* Define as 1 to use a boxed representation for all objects */
#define ONLY_BOXED_OBJECTS 0

/* Define as 1 to suspend non-deterministic computations */
#define YIELD_NONDET 0

/* Define as 1 to disable occurs check in unification */
#define NO_OCCURS_CHECK 0

#endif /* _CONFIG_H */
\nwnotused{config.h.in}\nwendcode{}

\nwixlogsorted{c}{{config.h.in}{NW3YYKtn-2GRKeP-1}{\nwixd{NW3YYKtn-2GRKeP-1}}}%
\nwixlogsorted{c}{{curry.h}{NW3YYKtn-2NNFqs-1}{\nwixd{NW3YYKtn-2NNFqs-1}}}%

