\nwfilename{refs.nw}\nwbegindocs{0}% -*- noweb-code-mode: c-mode -*-% ===> this file was generated automatically by noweave --- better not edit it
% $Id: refs.nw,v 2.1 2004/03/22 14:36:43 wlux Exp $
%
% Copyright (c) 2004, Wolfgang Lux
% See ../LICENSE for the full license.
%
\subsection{Mutable references}
The \texttt{IOExts} library implements mutable references in the
\texttt{IO} monad. The implementation is based on three primitives
that are defined in this file.

\nwenddocs{}\nwbegincode{1}\sublabel{NW1zFxc5-2LjkGD-1}\nwmargintag{{\nwtagstyle{}\subpageref{NW1zFxc5-2LjkGD-1}}}\moddef{refs.c~{\nwtagstyle{}\subpageref{NW1zFxc5-2LjkGD-1}}}\endmoddef\nwstartdeflinemarkup\nwprevnextdefs{\relax}{NW1zFxc5-2LjkGD-2}\nwenddeflinemarkup
#include "config.h"
#include <stdio.h>
#include <stdlib.h>
#include <string.h>
#include "debug.h"
#include "run.h"
#include "heap.h"
#include "stack.h"
#include "trail.h"
#include "eval.h"
#include "threads.h"
#include "trace.h"
#include "cam.h"

enum \{ IOREF_TAG \};

struct ioref_node \{
    NodeInfo *info;
    Node     *ref;
\};

#define ioref_node_size wordsof(struct ioref_node)

static NodeInfo ioref_info = \{
    IOREF_TAG, ioref_node_size, (const int *)0, (Label)eval_whnf, "IORef",
    (FinalFun)0
\};

\nwalsodefined{\\{NW1zFxc5-2LjkGD-2}\\{NW1zFxc5-2LjkGD-3}\\{NW1zFxc5-2LjkGD-4}}\nwnotused{refs.c}\nwendcode{}\nwbegindocs{2}\nwdocspar
The function {\Tt{}newIORef\nwendquote} creates a new {\Tt{}IORef\nwendquote} and initializes it
with the specified value.

\nwenddocs{}\nwbegincode{3}\sublabel{NW1zFxc5-2LjkGD-2}\nwmargintag{{\nwtagstyle{}\subpageref{NW1zFxc5-2LjkGD-2}}}\moddef{refs.c~{\nwtagstyle{}\subpageref{NW1zFxc5-2LjkGD-1}}}\plusendmoddef\nwstartdeflinemarkup\nwprevnextdefs{NW1zFxc5-2LjkGD-1}{NW1zFxc5-2LjkGD-3}\nwenddeflinemarkup
DECLARE_ENTRYPOINT(__newIORef);

FUNCTION(__newIORef)
\{
    struct ioref_node *ref;

    EXPORT_LABEL(__newIORef)
 ENTRY_LABEL(__newIORef)
    CHECK_HEAP(ioref_node_size);
    ref       = (struct ioref_node *)hp;
    ref->info = &ioref_info;
    ref->ref  = sp[0];
    hp       += ioref_node_size;
    
    sp += 2;
    RETURN((Node *)ref);
\}

\nwendcode{}\nwbegindocs{4}\nwdocspar
The function {\Tt{}readIORef\nwendquote} returns the value that is currently saved
in a mutable reference. Note that the value is \emph{not} evaluated
to head normal form because the \texttt{IO} monad is lazy in Curry
(as it is in Haskell).

\nwenddocs{}\nwbegincode{5}\sublabel{NW1zFxc5-2LjkGD-3}\nwmargintag{{\nwtagstyle{}\subpageref{NW1zFxc5-2LjkGD-3}}}\moddef{refs.c~{\nwtagstyle{}\subpageref{NW1zFxc5-2LjkGD-1}}}\plusendmoddef\nwstartdeflinemarkup\nwprevnextdefs{NW1zFxc5-2LjkGD-2}{NW1zFxc5-2LjkGD-4}\nwenddeflinemarkup
DECLARE_ENTRYPOINT(__readIORef);

FUNCTION(__readIORef)
\{
    struct ioref_node *ref;

   EXPORT_LABEL(__readIORef)
 ENTRY_LABEL(__readIORef)
    EVAL_RIGID(__readIORef);
    ASSERT(node_tag(sp[0]) == IOREF_TAG);
    ref = (struct ioref_node *)sp[0];
    sp += 2;
    RETURN(ref->ref);
\}

\nwendcode{}\nwbegindocs{6}\nwdocspar
The function {\Tt{}writeIORef\nwendquote} updates the value saved in a mutable
reference. The old value is saved on the trail even though the
\texttt{IO} monad is expected to run only deterministic code. However,
trailing prevents program crashes when this function is used inside an
{\Tt{}unsafePerformIO\nwendquote}. Note that this makes it impossible to use mutable
references for generating globally unique indices that are different
even across different branches of a non-deterministic search.

\nwenddocs{}\nwbegincode{7}\sublabel{NW1zFxc5-2LjkGD-4}\nwmargintag{{\nwtagstyle{}\subpageref{NW1zFxc5-2LjkGD-4}}}\moddef{refs.c~{\nwtagstyle{}\subpageref{NW1zFxc5-2LjkGD-1}}}\plusendmoddef\nwstartdeflinemarkup\nwprevnextdefs{NW1zFxc5-2LjkGD-3}{\relax}\nwenddeflinemarkup
DECLARE_ENTRYPOINT(__writeIORef);

FUNCTION(__writeIORef)
\{
    struct ioref_node *ref;

    EXPORT_LABEL(__writeIORef)
 ENTRY_LABEL(__writeIORef)
    EVAL_RIGID(__writeIORef);
    ASSERT(node_tag(sp[0]) == IOREF_TAG);
    ref = (struct ioref_node *)sp[0];
    SAVE(ref, ref);
    ref->ref = sp[1];
    sp += 3;
    RETURN(unit);
\}
\nwendcode{}

\nwixlogsorted{c}{{refs.c}{NW1zFxc5-2LjkGD-1}{\nwixd{NW1zFxc5-2LjkGD-1}\nwixd{NW1zFxc5-2LjkGD-2}\nwixd{NW1zFxc5-2LjkGD-3}\nwixd{NW1zFxc5-2LjkGD-4}}}%

