\nwfilename{unify.nw}\nwbegindocs{0}% -*- noweb-code-mode: c-mode -*-% ===> this file was generated automatically by noweave --- better not edit it
% $Id: unify.nw,v 2.15 2004/05/02 09:17:27 wlux Exp $
%
% Copyright (c) 2001-2004, Wolfgang Lux
% See ../LICENSE for the full license.
%
\subsection{Unification}\label{sec:unification}
The equality constraint \texttt{=:=} tries to unify its arguments.

\nwenddocs{}\nwbegincode{1}\sublabel{NW4bJadf-3IYIGE-1}\nwmargintag{{\nwtagstyle{}\subpageref{NW4bJadf-3IYIGE-1}}}\moddef{unify.h~{\nwtagstyle{}\subpageref{NW4bJadf-3IYIGE-1}}}\endmoddef\nwstartdeflinemarkup\nwenddeflinemarkup
extern Label bind_var(Node *, Node *, Label);

DECLARE_ENTRYPOINT(___61__58__61_);

#if NO_OCCURS_CHECK
# define occurs(var,arg)        false
#else
extern boolean occurs(Node *, Node *);
#endif

\nwnotused{unify.h}\nwendcode{}\nwbegindocs{2}\nwdocspar
\nwenddocs{}\nwbegincode{3}\sublabel{NW4bJadf-g2wII-1}\nwmargintag{{\nwtagstyle{}\subpageref{NW4bJadf-g2wII-1}}}\moddef{unify.c~{\nwtagstyle{}\subpageref{NW4bJadf-g2wII-1}}}\endmoddef\nwstartdeflinemarkup\nwprevnextdefs{\relax}{NW4bJadf-g2wII-2}\nwenddeflinemarkup
#include "config.h"
#include "debug.h"
#include <stdio.h>
#include <stdlib.h>
#include <string.h>
#include "run.h"
#include "heap.h"
#include "stack.h"
#include "eval.h"
#include "threads.h"
#include "spaces.h"
#include "trail.h"
#include "unify.h"
#include "cam.h"
#include "trace.h"

#define pair_cons_node_size constr_node_size(3)
static
NodeInfo pair_cons_info = \{
    CONS_TAG, pair_cons_node_size, (const int *)0, (Label)eval_whnf,
    ",:", (FinalFun)0
\};

DECLARE_LABEL(___61__58__61__1);
DECLARE_LABEL(___61__58__61__2);
DECLARE_LABEL(unify_data);
DECLARE_LABEL(unify_var);
DECLARE_LABEL(unify_args);
DECLARE_LABEL(unify_args_1);
DECLARE_LABEL(unify_args_2);


FUNCTION(___61__58__61_)
\{
    EXPORT_LABEL(___61__58__61_)
 ENTRY_LABEL(___61__58__61_)

    TRACE(("%I enter =:=%V\\n", 2, sp));
    GOTO(___61__58__61__1);
\}

static
FUNCTION(___61__58__61__1)
\{
    Node *aux;

 ENTRY_LABEL(___61__58__61__1)
    EVAL_FLEX_POLY(___61__58__61__1);
    aux   = sp[0];
    sp[0] = sp[1];
    sp[1] = aux;
    GOTO(___61__58__61__2);
\}

static
FUNCTION(___61__58__61__2)
\{
    unsigned int n;
    double       d, e;
    Node         *arg1, *arg2;

 ENTRY_LABEL(___61__58__61__2)
    EVAL_FLEX_POLY(___61__58__61__2);

    arg1 = sp[1];
    arg2 = sp[0];

    while ( is_boxed(arg1) && is_indir_node(arg1) )
        arg1 = arg1->n.node;
    if ( is_boxed(arg1) && is_variable_node(arg1) )
    \{
        /* check for trivial unification */
        if ( arg1 != arg2 )
        \{
            sp[0] = arg1;
            sp[1] = arg2;
            GOTO(unify_var);
        \}
    \}
    else if ( is_boxed(arg2) && is_variable_node(arg2) )
    \{
        sp[0] = arg2;
        sp[1] = arg1;
        GOTO(unify_var);
    \}
#if !ONLY_BOXED_OBJECTS
    else if ( is_unboxed(arg1) )
    \{
        ASSERT(is_unboxed(arg2));
        if ( arg1 != arg2 )
            FAIL();
    \}
#endif
    else
    \{
        ASSERT(is_boxed(arg2));
        if ( node_tag(arg1) != node_tag(arg2) )
            FAIL();
        switch ( node_tag(arg1) )
        \{
        case CHAR_TAG:
            if ( arg1->ch.ch != arg2->ch.ch )
                FAIL();
            break;
#if ONLY_BOXED_OBJECTS
        case INT_TAG:
            if ( arg1->i.i != arg2->i.i )
                FAIL();
            break;
#endif
        case FLOAT_TAG:
            get_float_val(d, arg1->f);
            get_float_val(e, arg2->f);
            if ( d != e )
                FAIL();
            break;

        case PAPP_TAG:
            if ( arg1->info != arg2->info )
                FAIL();
            if ( closure_argc(arg1) > 0 )
            \{
                sp[0] = arg1;
                sp[1] = arg2;
                GOTO(unify_data);
            \}
            break;

        case SEARCH_CONT_TAG:
            if ( arg1 != arg2 )
                FAIL();
            break;

        default:
            ASSERT(is_constr_node(arg1) || is_abstract_node(arg1));
            if ( is_abstract_node(arg1) )
            \{
                if ( arg1 != arg2 )
                    FAIL();
                n = 0;
            \}
            else if ( is_vector(arg1) )
            \{
                if ( arg1->a.length != arg2->a.length )
                    FAIL();
                n = vector_argc(arg1);
            \}
            else
                n = constr_argc(arg1);
            if ( n > 0 )
            \{
                sp[0] = arg1;
                sp[1] = arg2;
                GOTO(unify_data);
            \}
        \}
    \}

    sp += 2;
    RETURN(Success);
\}

\nwalsodefined{\\{NW4bJadf-g2wII-2}\\{NW4bJadf-g2wII-3}\\{NW4bJadf-g2wII-4}\\{NW4bJadf-g2wII-5}}\nwnotused{unify.c}\nwendcode{}\nwbegindocs{4}\nwdocspar
When two data constructors with arguments are unified and have the
same root, the unification must be applied -- concurrently -- to the
arguments. This is handled by {\Tt{}unify{\_}data\nwendquote}, which constructs a list
of all pairs of arguments. As a little optimization, {\Tt{}unify{\_}data\nwendquote}
already handles trivial unifications.

We consider a unification trivial here if both corresponding arguments
are ground atoms. In addition, if both arguments are data constructors
with different roots, the unification will fail immediately.

The argument list itself is processed by the function {\Tt{}unify{\_}args\nwendquote}
below.

\nwenddocs{}\nwbegincode{5}\sublabel{NW4bJadf-g2wII-2}\nwmargintag{{\nwtagstyle{}\subpageref{NW4bJadf-g2wII-2}}}\moddef{unify.c~{\nwtagstyle{}\subpageref{NW4bJadf-g2wII-1}}}\plusendmoddef\nwstartdeflinemarkup\nwprevnextdefs{NW4bJadf-g2wII-1}{NW4bJadf-g2wII-3}\nwenddeflinemarkup
static
FUNCTION(unify_data)
\{
    boolean      is_vect;
    unsigned int i, n;
    double       d, e;
    Node         *next, *arglist, *x, *y, **argp1, **argp2;

 ENTRY_LABEL(unify_data)

    is_vect = is_vector(sp[0]);
    n       = is_vect ? vector_argc(sp[0]) : constr_argc(sp[0]);

    CHECK_HEAP(n * pair_cons_node_size);
    argp1 = is_vect ? sp[0]->a.args : sp[0]->c.args;
    argp2 = is_vect ? sp[1]->a.args : sp[1]->c.args;

    arglist = nil;
    argp1  += n;
    argp2  += n;
    for ( i = n; i-- > 0; )
    \{
        x = *--argp1;
        y = *--argp2;
        while ( is_boxed(x) && is_indir_node(x) )
            x = x->n.node;
        while ( is_boxed(y) && is_indir_node(y) )
            y = y->n.node;

#if !ONLY_BOXED_OBJECTS
        if ( is_unboxed(x) )
        \{
            if ( is_unboxed(y) )
            \{
                if ( x != y )
                    FAIL();
                continue;
            \}
        \}
        else
#endif /* !ONLY_BOXED_OBJECTS */
            switch ( node_tag(x) )
            \{
            case CHAR_TAG:
                if ( is_char_node(y) )
                \{
                    if ( x->ch.ch != y->ch.ch )
                        FAIL();
                    continue;
                \}
                break;
#if ONLY_BOXED_OBJECTS
            case INT_TAG:
                if ( is_int_node(y) )
                \{
                    if ( int_val(x) != int_val(y) )
                        FAIL();
                    continue;
                \}
                break;
#endif /* ONLY_BOXED_OBJECTS */
            case FLOAT_TAG:
                if ( is_float_node(y) )
                \{
                    get_float_val(d, x->f);
                    get_float_val(e, y->f);
                    if ( d != e )
                        FAIL();
                    continue;
                \}
                break;
            case VARIABLE_TAG:
                if ( x == y )
                    continue;
                break;
            case PAPP_TAG:
                if ( is_papp_node(y) )
                \{
                    if ( x->info != y->info )
                        FAIL();
                    if ( closure_argc(x) == 0 )
                        continue;
                \}
                break;
            case SEARCH_CONT_TAG:
                if ( is_search_cont_node(y) )
                \{
                    if ( x != y )
                        FAIL();
                    continue;
                \}
                break;
            case CLOSURE_TAG:
            case SUSPEND_TAG:
            case QUEUEME_TAG:
                break;
            default:
                ASSERT(is_constr_node(x) || is_abstract_node(x));
                if ( is_constr_node(y) || is_abstract_node(y) )
                \{
                    if ( node_tag(x) != node_tag(y) )
                        FAIL();
                    if ( is_abstract_node(x) )
                    \{
                        if ( x != y )
                            FAIL();
                        continue;
                    \}
                    else if ( is_vector(x) )
                    \{
                        if ( x->a.length != y->a.length )
                            FAIL();
                        if ( vector_argc(x) == 0 )
                            continue;
                    \}
                    else if ( constr_argc(x) == 0 )
                        continue;
                \}
                break;
            \}

        next            = (Node *)hp;
        next->c.info    = &pair_cons_info;
        next->c.args[0] = x;
        next->c.args[1] = y;
        next->c.args[2] = arglist;
        arglist         = next;
        hp             += pair_cons_node_size;
    \}

    if ( arglist != nil )
    \{
        *++sp = arglist;
        GOTO(unify_args);
    \}
    sp += 2;
    RETURN(Success);
\}

\nwendcode{}\nwbegindocs{6}\nwdocspar
The unification of the arguments of data terms proceeds
concurrently. Instead of calling the code for the predefined operator
{\Tt{}{\&}\nwendquote}, we handle the concurrent evaluation ourselves. Thus, we can
make use of the fact that the unification always returns the value
{\Tt{}Success\nwendquote} and can never return an unbound variable.

\nwenddocs{}\nwbegincode{7}\sublabel{NW4bJadf-g2wII-3}\nwmargintag{{\nwtagstyle{}\subpageref{NW4bJadf-g2wII-3}}}\moddef{unify.c~{\nwtagstyle{}\subpageref{NW4bJadf-g2wII-1}}}\plusendmoddef\nwstartdeflinemarkup\nwprevnextdefs{NW4bJadf-g2wII-2}{NW4bJadf-g2wII-4}\nwenddeflinemarkup
static
FUNCTION(unify_args)
\{
    Node *susp, *arglist;

 ENTRY_LABEL(unify_args)
    CHECK_STACK(6);
    CHECK_HEAP(queueMe_node_size);

    arglist = sp[0];
    ASSERT(arglist->info == &pair_cons_info);

    if ( arglist->c.args[2] == nil )
    \{
        sp   -= 1;
        sp[0] = arglist->c.args[0];
        sp[1] = arglist->c.args[1];
        GOTO(___61__58__61_);
    \}

    susp        = (Node *)hp;
    susp->info  = &queueMe_info;
    susp->q.wq  = (ThreadQueue)0;
    susp->q.spc = ss;
    hp         += queueMe_node_size;

    sp   -= 6;
    sp[0] = arglist->c.args[0];
    sp[1] = arglist->c.args[1];
    sp[2] = (Node *)update;
    sp[3] = susp;
    sp[4] = (Node *)unify_args_1;
    sp[5] = susp;
    sp[6] = arglist->c.args[2];
    start_thread(5);
    GOTO(___61__58__61_);
\}

static
FUNCTION(unify_args_1)
\{
    Node *r;

 ENTRY_LABEL(unify_args_1)
    for ( r = sp[0]; node_tag(r) == INDIR_TAG; r = r->n.node )
        ;

    if ( node_tag(r) == SUCCESS_TAG )
        sp++;
    else
    \{
        ASSERT(node_tag(r) == QUEUEME_TAG);
        CHECK_STACK1();
        sp   -= 1;
        sp[0] = sp[2];
        sp[1] = (Node *)unify_args_2;
        sp[2] = r;
    \}
    GOTO(unify_args);
\}

static
FUNCTION(unify_args_2)
\{
    Node *r;

 ENTRY_LABEL(unify_args_2)
    ASSERT(node_tag(sp[0]) == SUCCESS_TAG);

    for ( r = sp[1]; node_tag(r) == INDIR_TAG; r = r->n.node )
        ;
    if ( node_tag(r) == QUEUEME_TAG )
    \{
        *++sp = r;
        GOTO(r->info->eval);
    \}
    ASSERT(node_tag(r) == SUCCESS_TAG);

    sp += 2;
    RETURN(r);
\}

\nwendcode{}\nwbegindocs{8}\nwdocspar
If a variable is unified with a data term, we have to perform an
occurs check in order to ensure that the term remains finite. The
check can be disabled with the \texttt{--disable-occurs-check}
configuration option.

As for {\Tt{}unify{\_}data\nwendquote}, we perform trivial unifications directly in
{\Tt{}unify{\_}var\nwendquote} and thus can avoid creating redundant threads and
variables. In fact, the code below works the other way around. First,
a flat copy of the data term is created and then those arguments whose
unification is not trivial are replaced by fresh variables.

\ToDo{The case of trivial unifications could be extended to all
ground data terms. In order to detect such constants the compiler
should add a mark to them.}

\nwenddocs{}\nwbegincode{9}\sublabel{NW4bJadf-g2wII-4}\nwmargintag{{\nwtagstyle{}\subpageref{NW4bJadf-g2wII-4}}}\moddef{unify.c~{\nwtagstyle{}\subpageref{NW4bJadf-g2wII-1}}}\plusendmoddef\nwstartdeflinemarkup\nwprevnextdefs{NW4bJadf-g2wII-3}{NW4bJadf-g2wII-5}\nwenddeflinemarkup
#if !NO_OCCURS_CHECK
boolean
occurs(Node *var, Node *arg)
\{
    boolean  is_vect;
    unsigned i, argc;
    Node     **argp;

    while ( is_boxed(arg) && node_tag(arg) == INDIR_TAG )
        arg = arg->n.node;
    if ( is_boxed(arg) )
    \{
        if ( arg == var )
            return true;
        if ( is_constr_node(arg) || is_papp_node(arg) )
        \{
            is_vect = is_vector(arg);
            argc    = is_vect ? vector_argc(arg) : constr_argc(arg);
            argp    = is_vect ? arg->a.args : arg->c.args;

            for ( i = 0; i < argc; i++ )
                if ( occurs(var, argp[i]) )
                    return true;
        \}
    \}
    return false;
\}
#endif /* !NO_OCCURS_CHECK */

static
FUNCTION(unify_var)
\{
    boolean      is_vect;
    unsigned int i, n, sz;
    Node         *var, *arg, *next, *arglist, **argp;
    Label        ret_ip;

 ENTRY_LABEL(unify_var)
    if ( !is_local_space(sp[0]->v.spc) )
    \{
        if ( is_boxed(sp[1]) && is_variable_node(sp[1])
             && is_local_space(sp[1]->v.spc) )
        \{
            next  = sp[0];
            sp[0] = sp[1];
            sp[1] = next;
        \}
        else
            GOTO(delay_thread(___61__58__61__1, sp[0]));
    \}

    var = sp[0];
    arg = sp[1];
    if ( occurs(var, arg) )
        FAIL();

    arglist = nil;
    if ( is_boxed(arg) && (is_constr_node(arg) || is_papp_node(arg)) )
    \{
        is_vect = is_vector(arg);
        n       = is_vect ? vector_argc(arg) : constr_argc(arg);

        if ( n > 0 )
        \{
            sz = is_vect ? vector_node_size(n) : constr_node_size(n);
            CHECK_HEAP(sz + n * (variable_node_size + pair_cons_node_size));
            memcpy(hp, sp[1], sz * word_size);
            sp[1] = (Node *)hp;
            hp   += sz;

            argp  = is_vect ? sp[1]->a.args : sp[1]->c.args;
            argp += n;
            for ( i = n; i-- > 0; )
            \{
                arg = *--argp;
                while ( is_boxed(arg) && is_indir_node(arg) )
                    arg = arg->n.node;
                /* Hack alert: Checking for !is_int_node(arg) first ensures
                 * that the remaining tests are applied to boxed nodes only */
                /* XXX avoid redundant unification of empty vectors */
                if ( !is_int_node(arg) && !is_float_node(arg)
                     && !is_variable_node(arg)
                     && !(is_constr_node(arg) && constr_argc(arg) == 0)
                     && !(is_papp_node(arg) && closure_argc(arg) == 0)
                     && !is_search_cont_node(arg)
                     && !is_abstract_node(arg) )
                \{
                    var = *argp     = (Node *)hp;
                    var->v.info     = &variable_info;
                    var->v.cstrs    = (Constraint *)0;
                    var->v.wq       = (ThreadQueue)0;
                    var->v.spc      = ss;
                    hp             += variable_node_size;
                    next            = (Node *)hp;
                    next->c.info    = &pair_cons_info;
                    next->c.args[0] = var;
                    next->c.args[1] = arg;
                    next->c.args[2] = arglist;
                    arglist         = next;
                    hp             += pair_cons_node_size;
                \}
                else
                    *argp = arg;
            \}
        \}
    \}

    /* bind the variable */
    var = sp[0];
    arg = sp[1];
    if ( arglist == nil )
    \{
        sp    += 2;
        ret_ip = (Label)sp[0];
        sp[0]  = Success;
        GOTO(bind_var(var, arg, ret_ip));
    \}
    *++sp = arglist;
    GOTO(bind_var(var, arg, unify_args));
\}

\nwendcode{}\nwbegindocs{10}\nwdocspar
When a variable is bound, the runtime system first has to check that
the binding does not conflict with any of the constraints imposed on
the variable. This is checked simply by calling the disequality
primitive for the bound value and each constraint in turn. The
variable node itself is overwritten with an indirection to the bound
value. Similar to the case of updating a suspended application node,
all threads that have been delayed by a rigid access to the variable
are woken again. In contrast to suspended applications, these threads
are run before continuing the current thread.

If a variable is bound to another variable, no thread is actually
woken. Instead, the wait-queues of both variables are concatenated. We
also must check the constraints of the other variable in this case.
As we cannot update a non-local variable, we must suspend the current
search until the non-local variable is instantiated if there are any
constraints or waiting threads on the bound variable.

In general, the {\Tt{}bind{\_}var\nwendquote} code is called immediately after a switch
instruction or from the unification code. However, when {\Tt{}bind{\_}var\nwendquote}
is called after resuming a search continuation, the variable may be
instantiated already. This may happen if a search strategy applies the
search continuation to some non-variable term in order to restrict the
search space as in the following example:
\begin{verbatim}
  main = concatMap try $ map (`inject` nonNull) $ try goal 
  goal xs = length xs =:= 1
  nonNull (_:_) = success
\end{verbatim}
In this case, {\Tt{}bind{\_}var\nwendquote} is implicitly transformed into a
unification between the two values.

\nwenddocs{}\nwbegincode{11}\sublabel{NW4bJadf-g2wII-5}\nwmargintag{{\nwtagstyle{}\subpageref{NW4bJadf-g2wII-5}}}\moddef{unify.c~{\nwtagstyle{}\subpageref{NW4bJadf-g2wII-1}}}\plusendmoddef\nwstartdeflinemarkup\nwprevnextdefs{NW4bJadf-g2wII-4}{\relax}\nwenddeflinemarkup
DECLARE_LABEL(check_constraints);
DECLARE_LABEL(check_constraints_1);
DECLARE_LABEL(wake);
DECLARE_LABEL(resume_bind);
DECLARE_LABEL(bind_var_1);

Label
bind_var(Node *var, Node *node, Label cont)
\{
    boolean     is_var;
    Constraint  *cstrs;
    ThreadQueue wq;

    if ( !is_boxed(var) || !is_variable_node(var)
         || !is_local_space(var->v.spc) )
    \{
        CHECK_STACK(4);
        sp   -= 4;
        sp[0] = var;
        sp[1] = node;
        sp[2] = (Node *)bind_var_1;
        sp[3] = (Node *)cont;
        return ___61__58__61_;
    \}

    cstrs  = var->v.cstrs;
    wq     = var->v.wq;
    is_var = false;
    for (;;)
    \{
        if ( is_boxed(node) )
            switch ( node->info->tag )
            \{
            case INDIR_TAG:
                node = node->n.node;
                continue;
            case VARIABLE_TAG:
                if ( !is_local_space(node->v.spc)
                     && (wq != (ThreadQueue)0 || cstrs != (Constraint *)0) )
                \{
                    CHECK_STACK(3);
                    sp  -= 3;
                    sp[0] = var;
                    sp[1] = node;
                    sp[2] = (Node *)cont;
                    return delay_thread(resume_bind, node);
                \}
                is_var = true;
                break;
            \}
        break;
    \}

    /* update the variable */
    TRACE(("%I %N = %N\\n", var, node));
    SAVE(var, v.cstrs);
    SAVE(var, v.wq);
    var->v.cstrs = (Constraint *)0;
    var->v.wq    = (ThreadQueue)0;
    var->info    = &variable_indir_info;
    var->n.node  = node;

    /* handle the waitqueue of the variable */
    if ( wq != (ThreadQueue)0 )
    \{
        if ( is_var )
        \{
            SAVE(node, v.wq);
            node->v.wq = join_queues(wq, node->v.wq);
        \}
        else
        \{
            CHECK_STACK(2);
            sp   -= 2;
            sp[0] = (Node *)wq;
            sp[1] = (Node *)cont;
            cont  = wake;
        \}
    \}

    /* if there are any constraints on the variable re-check them */
    if ( cstrs != (Constraint *)0 )
    \{
        CHECK_STACK(3);
        sp   -= 3;
        sp[0] = (Node *)cstrs;
        sp[1] = node;
        sp[2] = (Node *)cont;
        cont  = check_constraints;
    \}

    /* we need to check the constraints of the other variable, too */
    if ( is_var && is_local_space(node->v.spc)
         && node->v.cstrs != (Constraint *)0 )
    \{
        CHECK_STACK(3);
        sp   -= 3;
        sp[0] = (Node *)node->v.cstrs;
        sp[1] = node;
        sp[2] = (Node *)cont;
        cont  = check_constraints;
        SAVE(node, v.cstrs);
        node->v.cstrs = (Constraint *)0;
    \}

    /* continue evaluation */
    return cont;
\}

static
FUNCTION(bind_var_1)
\{
    Label cont;

 ENTRY_LABEL(bind_var_1)
    cont = (Label)sp[1];
    sp  += 2;
    GOTO(cont);
\}

static
FUNCTION(resume_bind)
\{
    Node  *var, *node;
    Label cont;

 ENTRY_LABEL(resume_bind)
    var  = sp[0];
    node = sp[1];
    cont = (Label)sp[2];
    sp  += 3;
    GOTO(bind_var(var, node, cont));
\}

static
FUNCTION(check_constraints)
\{
    Node       *node;
    Constraint *cstrs;

 ENTRY_LABEL(check_constraints)
    cstrs = (Constraint *)sp[0];
    node  = sp[1];
    ASSERT(cstrs != (Constraint *)0);

    CHECK_STACK(3);
    sp   -= 3;
    sp[0] = node;
    sp[1] = (Node *)cstrs;
    sp[2] = (Node *)check_constraints_1;
    sp[3] = (Node *)cstrs->cstrs;
    GOTO(cstrs->info->eval);
\}

static
FUNCTION(check_constraints_1)
\{
    Constraint *cstrs;
    Label      ret_ip;

 ENTRY_LABEL(check_constraints_1)
    /* XXX check for suspended constraints? */
    cstrs = (Constraint *)sp[1];
    if ( cstrs != (Constraint *)0 )
    \{
        sp += 1;
        GOTO(check_constraints);
    \}

    ret_ip = (Label)sp[3];
    sp    += 4;
    GOTO(ret_ip);
\}

static
FUNCTION(wake)
\{
    Label       ret_ip;
    ThreadQueue wq;

 ENTRY_LABEL(wake)
    /* handle the waitqueue of the variable */
    wq     = (ThreadQueue)sp[0];
    ret_ip = (Label)sp[1];
    sp    += 2;

    /* wake all threads from the queue */
    GOTO(activate_threads(wq, ret_ip));
\}
\nwendcode{}

\nwixlogsorted{c}{{unify.c}{NW4bJadf-g2wII-1}{\nwixd{NW4bJadf-g2wII-1}\nwixd{NW4bJadf-g2wII-2}\nwixd{NW4bJadf-g2wII-3}\nwixd{NW4bJadf-g2wII-4}\nwixd{NW4bJadf-g2wII-5}}}%
\nwixlogsorted{c}{{unify.h}{NW4bJadf-3IYIGE-1}{\nwixd{NW4bJadf-3IYIGE-1}}}%

