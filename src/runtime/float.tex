\nwfilename{float.nw}\nwbegindocs{0}% -*- noweb-code-mode: c-mode -*-% ===> this file was generated automatically by noweave --- better not edit it
% $Id: float.nw,v 2.3 2003/06/08 20:38:13 wlux Exp $
%
% Copyright (c) 2002, Wolfgang Lux
% See ../LICENSE for the full license.
%
\subsection{Floating-point primitives}
The primitive arithmetic operations for floating-point numbers as well
as the conversion functions between floating-point numbers and
integers are implemented in this file.

\nwenddocs{}\nwbegincode{1}\sublabel{NW1fcbZv-3irFiT-1}\nwmargintag{{\nwtagstyle{}\subpageref{NW1fcbZv-3irFiT-1}}}\moddef{float.c~{\nwtagstyle{}\subpageref{NW1fcbZv-3irFiT-1}}}\endmoddef\nwstartdeflinemarkup\nwprevnextdefs{\relax}{NW1fcbZv-3irFiT-2}\nwenddeflinemarkup
#include "config.h"
#include <stdio.h>
#include <stdlib.h>
#include <math.h>
#include "debug.h"
#include "run.h"
#include "heap.h"
#include "stack.h"
#include "eval.h"
#include "threads.h"
#include "cam.h"
#include "trace.h"

\nwalsodefined{\\{NW1fcbZv-3irFiT-2}\\{NW1fcbZv-3irFiT-3}}\nwnotused{float.c}\nwendcode{}\nwbegindocs{2}\nwdocspar
The implementation of the integer operations is straight
forward. After evaluating both arguments the operation is
performed. Its implementation depends somewhat on whether we are using
unboxed or boxed integers.

\nwenddocs{}\nwbegincode{3}\sublabel{NW1fcbZv-3irFiT-2}\nwmargintag{{\nwtagstyle{}\subpageref{NW1fcbZv-3irFiT-2}}}\moddef{float.c~{\nwtagstyle{}\subpageref{NW1fcbZv-3irFiT-1}}}\plusendmoddef\nwstartdeflinemarkup\nwprevnextdefs{NW1fcbZv-3irFiT-1}{NW1fcbZv-3irFiT-3}\nwenddeflinemarkup
DECLARE_ENTRYPOINT(___43__46_);
DECLARE_LABEL(___43__46__1);

FUNCTION(___43__46_)
\{
    Node *aux;

    EXPORT_LABEL(___43__46_)
 ENTRY_LABEL(___43__46_)
    EVAL_RIGID_FLOAT(___43__46_);
    aux   = sp[0];
    sp[0] = sp[1];
    sp[1] = aux;
    GOTO(___43__46__1);
\}

static
FUNCTION(___43__46__1)
\{
    double d, e;
    Node   *r;

 ENTRY_LABEL(___43__46__1)
    EVAL_RIGID_FLOAT(___43__46__1);
    get_float_val(d, sp[1]->f);
    get_float_val(e, sp[0]->f);
    sp += 2;

    CHECK_HEAP(float_node_size);
    r       = (Node *)hp;
    r->info = &float_info;
    put_float_val(r->f, d + e);
    hp += float_node_size;
    RETURN(r);
\}


DECLARE_ENTRYPOINT(___45__46_);
DECLARE_LABEL(___45__46__1);

FUNCTION(___45__46_)
\{
    Node *aux;

    EXPORT_LABEL(___45__46_)
 ENTRY_LABEL(___45__46_)
    EVAL_RIGID_FLOAT(___45__46_);
    aux   = sp[0];
    sp[0] = sp[1];
    sp[1] = aux;
    GOTO(___45__46__1);
\}

static
FUNCTION(___45__46__1)
\{
    double d, e;
    Node   *r;

 ENTRY_LABEL(___45__46__1)
    EVAL_RIGID_FLOAT(___45__46__1);
    get_float_val(d, sp[1]->f);
    get_float_val(e, sp[0]->f);
    sp += 2;

    CHECK_HEAP(float_node_size);
    r       = (Node *)hp;
    r->info = &float_info;
    put_float_val(r->f, d - e);
    hp += float_node_size;
    RETURN(r);
\}


DECLARE_ENTRYPOINT(___42__46_);
DECLARE_LABEL(___42__46__1);

FUNCTION(___42__46_)
\{
    Node *aux;

    EXPORT_LABEL(___42__46_)
 ENTRY_LABEL(___42__46_)
    EVAL_RIGID_FLOAT(___42__46_);
    aux   = sp[0];
    sp[0] = sp[1];
    sp[1] = aux;
    GOTO(___42__46__1);
\}

static
FUNCTION(___42__46__1)
\{
    double d, e;
    Node   *r;

 ENTRY_LABEL(___42__46__1)
    EVAL_RIGID_FLOAT(___42__46__1);
    get_float_val(d, sp[1]->f);
    get_float_val(e, sp[0]->f);
    sp += 2;

    CHECK_HEAP(float_node_size);
    r       = (Node *)hp;
    r->info = &float_info;
    put_float_val(r->f, d * e);
    hp += float_node_size;
    RETURN(r);
\}

DECLARE_ENTRYPOINT(___47__46_);
DECLARE_LABEL(___47__46__1);

FUNCTION(___47__46_)
\{
    Node *aux;

    EXPORT_LABEL(___47__46_)
 ENTRY_LABEL(___47__46_)
    EVAL_RIGID_FLOAT(___47__46_);
    aux   = sp[0];
    sp[0] = sp[1];
    sp[1] = aux;
    GOTO(___47__46__1);
\}

static
FUNCTION(___47__46__1)
\{
    double d, e;
    Node   *r;

 ENTRY_LABEL(___47__46__1)
    EVAL_RIGID_FLOAT(___47__46__1);
    get_float_val(d, sp[1]->f);
    get_float_val(e, sp[0]->f);
    sp += 2;

    CHECK_HEAP(float_node_size);
    r       = (Node *)hp;
    r->info = &float_info;
    put_float_val(r->f, d / e);
    hp += float_node_size;
    RETURN(r);
\}

\nwendcode{}\nwbegindocs{4}\nwdocspar
In case of the conversion operations we have to handle unboxed and
boxed integer numbers differently.

\nwenddocs{}\nwbegincode{5}\sublabel{NW1fcbZv-3irFiT-3}\nwmargintag{{\nwtagstyle{}\subpageref{NW1fcbZv-3irFiT-3}}}\moddef{float.c~{\nwtagstyle{}\subpageref{NW1fcbZv-3irFiT-1}}}\plusendmoddef\nwstartdeflinemarkup\nwprevnextdefs{NW1fcbZv-3irFiT-2}{\relax}\nwenddeflinemarkup
DECLARE_ENTRYPOINT(__floatFromInt);

FUNCTION(__floatFromInt)
\{
    long i;
    Node *r;

    EXPORT_LABEL(__floatFromInt)
 ENTRY_LABEL(__floatFromInt)
    EVAL_RIGID_INT(__floatFromInt);
    i   = int_val(sp[0]);
    sp += 1;

    CHECK_HEAP(float_node_size);
    r       = (Node *)hp;
    r->info = &float_info;
    put_float_val(r->f, i);
    hp += float_node_size;
    RETURN(r);
\}


DECLARE_ENTRYPOINT(__truncateFloat);

FUNCTION(__truncateFloat)
\{
    double d;
    Node   *r;

    EXPORT_LABEL(__truncateFloat)
 ENTRY_LABEL(__truncateFloat)
    EVAL_RIGID_FLOAT(__truncateFloat);
    get_float_val(d, sp[0]->f);
    sp += 1;

#if ONLY_BOXED_OBJECTS
    CHECK_HEAP(int_node_size);
    r       = (Node *)hp;
    r->info = &int_info;
    r->i.i  = (int)d;
    hp     += int_node_size;
#else
    r = mk_int((int)d);
#endif
    RETURN(r);
\}


DECLARE_ENTRYPOINT(__roundFloat);

FUNCTION(__roundFloat)
\{
    double d, frac;
    Node   *r;

    EXPORT_LABEL(__roundFloat)
 ENTRY_LABEL(__roundFloat)
    EVAL_RIGID_FLOAT(__roundFloat);
    get_float_val(d, sp[0]->f);
    sp += 1;
#define odd(n) (n & 0x01)
    frac = modf(d, &d);
    if ( frac > 0.5 || (frac == 0.5 && odd((int)d)) )
        d += 1.0;
    else if ( frac < -0.5 || (frac == -0.5 && odd((int)d)) )
        d -= 1.0;
#undef odd

#if ONLY_BOXED_OBJECTS
    CHECK_HEAP(int_node_size);
    r       = (Node *)hp;
    r->info = &int_info;
    r->i.i  = (int)d;
    hp     += int_node_size;
#else
    r = mk_int((int)d);
#endif
    RETURN(r);
\}
\nwendcode{}

\nwixlogsorted{c}{{float.c}{NW1fcbZv-3irFiT-1}{\nwixd{NW1fcbZv-3irFiT-1}\nwixd{NW1fcbZv-3irFiT-2}\nwixd{NW1fcbZv-3irFiT-3}}}%

