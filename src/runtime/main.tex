\nwfilename{main.nw}\nwbegindocs{0}% -*- noweb-code-mode: c-mode -*-% ===> this file was generated automatically by noweave --- better not edit it
% $Id: main.nw,v 2.6 2003/06/26 12:24:56 wlux Exp $
%
% Copyright (c) 2001-2003, Wolfgang Lux
% See ../LICENSE for the full license.
%
\nwenddocs{}\nwbegindocs{1}\nwdocspar
\subsection{Runtime system initialization}
The {\Tt{}main\nwendquote} function of a Curry program must initialize the runtime
system by calling the function {\Tt{}curry{\_}init\nwendquote}. This function parses
the command line options and initializes the runtime system
accordingly. At the end, the program should call {\Tt{}curry{\_}terminate\nwendquote}
in order to display the statistics gathered during the run.

The main program is also responsible for providing default sizes for
the heap, the stack, and the trail. To this end it must define the
global variables {\Tt{}heapsize\nwendquote}, {\Tt{}stacksize\nwendquote}, and {\Tt{}trailsize\nwendquote}. The
actual sizes can be changed at runtime with the runtime system options
\texttt{-h}, \texttt{-k}, and \texttt{-t}, respectively. The defaults
sizes (4 MBytes for the heap and 512 kBytes for the stack and the
trail) are available with the preprocessor constants
{\Tt{}DEFAULT{\_}HEAPSIZE\nwendquote}, {\Tt{}DEFAULT{\_}STACKSIZE\nwendquote}, and
{\Tt{}DEFAULT{\_}TRAILSIZE\nwendquote} which can be redefined at compile time.

The sizes and also the base addresses of the heap, the stack, and the
trail are rounded to the next page boundary. The entry code therefore
saves the size of a memory page in the global variable {\Tt{}pagesize\nwendquote}.
If the page size cannot be determined from the operating system a
default of 4k is used. The value of {\Tt{}pagemask\nwendquote} is always set to
{\Tt{}pagesize\ -\ 1\nwendquote}.

\nwenddocs{}\nwbegincode{2}\sublabel{NW1Xx3lK-3j9QCn-1}\nwmargintag{{\nwtagstyle{}\subpageref{NW1Xx3lK-3j9QCn-1}}}\moddef{main.h~{\nwtagstyle{}\subpageref{NW1Xx3lK-3j9QCn-1}}}\endmoddef\nwstartdeflinemarkup\nwenddeflinemarkup
#define k 1024
#define M k*k

#ifndef DEFAULT_HEAPSIZE
# define DEFAULT_HEAPSIZE 4*M
#endif

#ifndef DEFAULT_STACKSIZE
# define DEFAULT_STACKSIZE 512*k
#endif
#ifndef DEFAULT_TRAILSIZE
# define DEFAULT_TRAILSIZE 512*k
#endif

#ifndef DEFAULT_SHOW_STATS
# define DEFAULT_SHOW_STATS 0
#endif

#ifndef DEFAULT_DO_TRACE
# define DEFAULT_DO_TRACE 0
#endif

extern unsigned int heapsize;
extern unsigned int stacksize;
extern unsigned int trailsize;

extern int do_trace;
extern int show_stats;

extern unsigned int pagesize;
extern unsigned int pagemask;

extern void curry_init(int *argc, char *argv[]);
extern void curry_terminate(void);

\nwnotused{main.h}\nwendcode{}\nwbegindocs{3}\nwdocspar
The function {\Tt{}curry{\_}init\nwendquote} extracts the runtime system options from
the command line and initializes the runtime system using these
options. Options to the runtime system must be enclosed by
\texttt{+RTS} and \texttt{-RTS}. The latter may be omitted if no
further arguments follow it on the command line. When {\Tt{}curry{\_}init\nwendquote}
returns, all runtime system options have been removed from the
argument vector.

The function {\Tt{}curry{\_}terminate\nwendquote} performs some clean up after the
program has finished. At present, this function only prints the
current statistics.

\nwenddocs{}\nwbegincode{4}\sublabel{NW1Xx3lK-1ezWyf-1}\nwmargintag{{\nwtagstyle{}\subpageref{NW1Xx3lK-1ezWyf-1}}}\moddef{main.c~{\nwtagstyle{}\subpageref{NW1Xx3lK-1ezWyf-1}}}\endmoddef\nwstartdeflinemarkup\nwprevnextdefs{\relax}{NW1Xx3lK-1ezWyf-2}\nwenddeflinemarkup
#include "config.h"
#include <stdio.h>
#include <stdlib.h>
#include <string.h>
#include <unistd.h>
#include <limits.h>
#include "run.h"
#include "heap.h"
#include "stack.h"
#include "trail.h"
#include "files.h"
#include "trace.h"
#include "main.h"
#include "stats.h"

static void         bad_option(void) __attribute__((noreturn));
static unsigned int parse_size(const char *, const char *);

#define DEFAULT_PAGESIZE 4*k

unsigned int pagesize, pagemask;

static void
bad_option()
\{
    fprintf(stderr, "Valid runtime system options:\\n");
    fprintf(stderr, " -b MODE  set buffer mode for standard input/output\\n");
    fprintf(stderr, "          valid MODEs: n  unbuffered\\n");
    fprintf(stderr, "                       l  line buffered\\n");
    fprintf(stderr, "                       f  fully buffered\\n");
    fprintf(stderr, " -d       trace program execution\\n");
    fprintf(stderr, " -h SIZE  set heap size to SIZE bytes (default: %d)\\n",
            heapsize);
    fprintf(stderr, " -p       print statistics at end of run\\n");
    fprintf(stderr, " -k SIZE  set stack size to SIZE bytes (default: %d)\\n",
            stacksize);
    fprintf(stderr, " -t SIZE  set trail size to SIZE bytes (default: %d)\\n",
            trailsize);
    exit(1);
\}

void
curry_init(int *p_argc, char *argv[])
\{
    boolean rts;
    char    *cp, *arg;
    int     i, j, argc;
    int     bufmode = -1, bufsize = 0;

    /* get system page size */
    pagesize = getpagesize();
    if ( pagesize == (unsigned)-1 )
        pagesize = DEFAULT_PAGESIZE;
    pagemask = pagesize - 1;

    /* process rts options */
    argc = *p_argc;
    rts  = false;
    for ( i = j = 1; i < argc; i++ )
    \{
        cp = argv[i];
        if ( !rts )
            if ( strcmp(cp, "+RTS") == 0 )
                rts = true;
            else
                argv[j++] = cp;
        else if ( strcmp(cp, "-RTS") == 0 )
            rts = false;
        else
        \{
            if ( *cp == '-' )
                for ( cp++; *cp != '\\0'; cp++ )
                    switch ( *cp )
                    \{
                    case 'd':
                        do_trace++;
                        break;
                    case 'p':
                        show_stats++;
                        break;
                    case 'b':
                    case 'h':
                    case 'k':
                    case 't':
                        if ( cp[1] != '\\0' )
                            arg = cp + 1;
                        else if ( ++i < argc )
                            arg = argv[i];
                        else
                        \{
                            fprintf(stderr, "%s: missing argument after -%c\\n",
                                    argv[0], *cp);
                            bad_option();
                        \}
                        switch ( *cp )
                        \{
                        case 'b':
                            if ( strcmp(arg, "n") == 0 )
                                bufmode = _IONBF;
                            else if ( strcmp(arg, "l") == 0 )
                                bufmode = _IOLBF;
                            else if ( strcmp(arg, "f") == 0 )
                                bufmode = _IOFBF;
                            else if ( *arg == 'f' )
                            \{
                                bufmode = _IOFBF;
                                bufsize = parse_size("buffer size", arg + 1);
                            \}
                            else
                            \{
                                fprintf(stderr,
                                        "%s: invalid file buffer mode %s\\n",
                                        argv[0], arg);
                                bad_option();
                            \}
                            break;
                        case 'h':
                            heapsize = parse_size("heap size", arg);
                            break;
                        case 'k':
                            stacksize = parse_size("stack size", arg);
                            break;
                        case 't':
                            trailsize = parse_size("trail size", arg);
                            break;
                        \}
                        cp = "\\0";
                        break;
                    default:
                        fprintf(stderr,
                                "%s: unknown runtime system option %c\\n",
                                argv[0], *cp);
                        bad_option();
                    \}
            else
            \{
                fprintf(stderr, "%s: unknown runtime system argument %s\\n",
                        argv[0], cp);
                bad_option();
            \}
        \}
    \}
    argv[j] = (char *)0;
    *p_argc = j;

    /* initialize runtime system */
    init_chars();
    init_stack(stacksize);
    init_trail(trailsize);
    init_heap(heapsize);
    init_files(bufmode, bufsize);
    stats_init(show_stats);
\}

void
curry_terminate()
\{
    stats_terminate(hp - heap_base);
\}

\nwalsodefined{\\{NW1Xx3lK-1ezWyf-2}}\nwnotused{main.c}\nwendcode{}\nwbegindocs{5}\nwdocspar
As a handy abbreviation, the function {\Tt{}parse{\_}size\nwendquote} parses numbers
with the suffixes {\Tt{}k\nwendquote} and {\Tt{}K\nwendquote} as kBytes and with the suffixes
{\Tt{}m\nwendquote} and {\Tt{}M\nwendquote} as MBytes.

\nwenddocs{}\nwbegincode{6}\sublabel{NW1Xx3lK-1ezWyf-2}\nwmargintag{{\nwtagstyle{}\subpageref{NW1Xx3lK-1ezWyf-2}}}\moddef{main.c~{\nwtagstyle{}\subpageref{NW1Xx3lK-1ezWyf-1}}}\plusendmoddef\nwstartdeflinemarkup\nwprevnextdefs{NW1Xx3lK-1ezWyf-1}{\relax}\nwenddeflinemarkup
static unsigned int
parse_size(const char *what, const char *arg)
\{
    long size;
    char *end;

    size = strtol(arg, &end, 0);
    if ( *end != '\\0' )
    \{
        if ( strcmp(end, "m") == 0 || strcmp(end, "M") == 0 )
        \{
            size = size > LONG_MAX / M ? LONG_MAX : size * M;
        \}
        else if ( strcmp(end, "k") == 0 || strcmp(end, "K") == 0 )
        \{
            size = size > LONG_MAX / k ? LONG_MAX : size * k;
        \}
        else
            size = -1;
    \}

    if ( size <= 0 )
    \{
        fprintf(stderr, "invalid %s: %s\\n", what, arg);
        exit(1);
    \}

    return size > LONG_MAX ? LONG_MAX : size;
\}
\nwendcode{}

\nwixlogsorted{c}{{main.c}{NW1Xx3lK-1ezWyf-1}{\nwixd{NW1Xx3lK-1ezWyf-1}\nwixd{NW1Xx3lK-1ezWyf-2}}}%
\nwixlogsorted{c}{{main.h}{NW1Xx3lK-3j9QCn-1}{\nwixd{NW1Xx3lK-3j9QCn-1}}}%

